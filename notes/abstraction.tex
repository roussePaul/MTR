\documentclass{article}

\usepackage{amsmath, amsfonts, amsthm}
\usepackage{nicefrac}
\usepackage{breqn}
\usepackage{tabularx}
\usepackage{stmaryrd} % for llbracket and rrbracket

\begin{document}
\title{Error in partially observed abstractions}
\author{Paul Rousse}
\maketitle



\newcommand{\mle}{\prec}
\newcommand{\mleq}{\preceq}
\newcommand{\minf}[1]{\underline{#1}}
\newcommand{\msup}[1]{\overline{#1}}

\newcommand{\vect}[1]{\mathbf{#1}}

\newcommand{\x}{\vect{x}}%
\renewcommand{\u}{\vect{u}}%
\newcommand{\w}{\vect{w}}%

\newcommand{\y}{\vect{y}}%
\newcommand{\yk}{\vect{y}_k}%
\newcommand{\ykn}{\vect{y}_{k+1}}%

\newcommand{\xk}{\vect{x}_k}%
\newcommand{\xkn}{\vect{x}_{k+1}}%

\newcommand{\uk}{\vect{u}_k}%
\newcommand{\wk}{\vect{w}_k}%

\newcommand{\xo}{\vect{x}^o}%
\newcommand{\xr}{\vect{x}^r}%

\newcommand{\Ao}{A_o}%
\newcommand{\Ar}{A_r}%
\newcommand{\Aro}{A_{ro}}%

\newcommand{\Bo}{B_o}%
\newcommand{\Br}{B_r}%

\newcommand{\Eo}{E_o}%
\newcommand{\Er}{E_r}%

\newcommand{\Xr}{X_r}%

\newcommand{\Xrinv}{\mathcal{X}_r}%
\newcommand{\xrinf}{\minf{\x}_r}%
\newcommand{\xrsup}{\msup{\x}_r}%


\newcommand{\Wsup}{\msup{W}}
\newcommand{\Winf}{\minf{W}}
\newcommand{\Wk}{W_k}

\renewcommand{\wr}{\vect{w}^r}
\newcommand{\Wrsup}{\msup{W}^r}
\newcommand{\Wrinf}{\minf{W}^r}
\newcommand{\Wrk}{W^r_k}

\newcommand{\traj}{\varphi}

\section{System definition}
We will take the following system:

\begin{equation}
\begin{aligned}
\xkn &=
\begin{bmatrix} \Ao&\Aro\\ 0& \Ar \end{bmatrix} \xk + 
\begin{bmatrix} \Bo\\ \Br \end{bmatrix} \uk + 
\begin{bmatrix} \Eo\\ \Er \end{bmatrix} \wk
\\
\yk &= \xo_k
\end{aligned}
\end{equation}

We will suppose that the system is monotonic. 
And that the noise and the inputs are bounded:
\begin{equation}
\begin{aligned}
\u &\in [\minf{\u},\msup{\u}]\\
\w &\in [\minf{\w},\msup{\w}]\\
\end{aligned}
\end{equation}

\section{Abstraction reduction}

The abstraction reduction is performed in this way. First, we define the smallest invariant of the unobserved state:
$$
\Xrinv = [\xrinf,\xrsup]
$$
where,
$$
\begin{aligned}
\xrinf &= (I-A)^{-1} (\Br \minf{\u} + \Er \minf{\w})\\
\xrsup &= (I-A)^{-1} (\Br \msup{\u} + \Er \msup{\w})
\end{aligned}
$$

Lets now define analytically the set $\Xr(U)$ that correspond to the interval that contain the image of $\Xrinv$ after applying the control sequence $U = [u_1,\dots,u_n]$.
Thanks to the monotonicity property, we have the following property:
\newcommand{\xui}{\minf{\x}^U_r}
\newcommand{\xus}{\msup{\x}^U_r}
\newcommand{\xu}{\x^U_r}
$$\Xr(U) = [\xui,\xus]$$
where
$$\xui = \traj (\xrinf,U,\Winf)
\textrm{, }
\xus = \traj (\xrsup,U,\Wsup)$$
$$ \Winf = [\minf{\w},\dots,\minf{\w}]
\textrm{, }
\Wsup = [\msup{\w},\dots,\msup{\w}]$$
so that $\Wsup$ and $\Winf$ are a sequence of $n$ elements.


\newcommand{\An}{\mathcal{A}^r_n}
\newcommand{\Bn}{\mathcal{B}^r_n}
\newcommand{\En}{\mathcal{E}^r_n}
\begin{align}
\xu &=
\begin{bmatrix}
A_r^k & A_r^{k-1} &\dots & A_r & I
\end{bmatrix}
\begin{bmatrix}
\x_0 \\
B_r \u_0 + E_r \w_0\\
\vdots \\
B_r \u_{k-1} + E_r \w_{k-1}\\
\end{bmatrix}\\
\xu &=
\An
\begin{bmatrix}
\x_0 \\
\Bn U + \En W
\end{bmatrix}
\end{align}

where
$$
\An = 
\begin{bmatrix}
A_r^n & A_r^{n-1} &\dots & A_r & I
\end{bmatrix}
\textrm{, }
\Bn = 
\begin{bmatrix}
B_r\\
\vdots\\
B_r
\end{bmatrix}
\textrm{ and }
\En = 
\begin{bmatrix}
E_r\\
\vdots\\
E_r
\end{bmatrix}
$$

\newcommand{\xuki}{\minf{\x}^{U_k}_r}
\newcommand{\xuks}{\msup{\x}^{U_k}_r}
\newcommand{\xuk}{\x^{U_k}_r}
Let define the variable $\xuki$ and $\xuks$ (guess it from previous definition).

\section{Admissible noise}
We would like now to compute the admissible noise that our new abstraction is able to handle.
\newcommand{\sykn}{\msup{\y}_{k+1}}%
\newcommand{\iykn}{\minf{\y}_{k+1}}%
Recall the definition of the alternate simulation relation.
$$
\ykn \in [\sykn,\iykn]
$$
where 
\begin{align*}
\sykn &= C \msup{\xo}_{k+1}\\
\sykn &= C A 
\begin{bmatrix}
\xo_{k} \\
\xuks
\end{bmatrix}
+ C B \uk + C E \msup{\w}
\\
\iykn &= C \minf{\xo}_{k+1}\\
\iykn &= C A 
\begin{bmatrix}
\xo_{k} \\
\xuki
\end{bmatrix}
+ C B \uk + C E \minf{\w}
\\
\ykn &= C \minf{\xo}_{k+1}\\
\ykn &= C A 
\begin{bmatrix}
\xo_{k} \\
\xuk
\end{bmatrix}
+ C B \uk + C E \wk
\end{align*}

We will now use the different form to get a condition over $\Wk$

Lets gooooo!

\begin{align*}
\ykn \mleq \sykn
& \Leftrightarrow
0 \mleq \sykn - \ykn \\
& \Leftrightarrow
0 \mleq
C A \begin{bmatrix} 0 \\ \xuks-\xuk \end{bmatrix}
+ C E (\msup{\w} -\wk)
\\
& \Leftrightarrow
0 \mleq
C A 
\begin{bmatrix} 0 \\ \An \end{bmatrix} 
\begin{bmatrix}
\xrsup - \xr_{k-n}\ \\
\En (\Wrsup - \Wrk)
\end{bmatrix}
+ C E (\msup{\w} -\wk)
\end{align*}

We have the following equality:
\begin{align*}
C A \begin{bmatrix} 0 \\ \An \end{bmatrix} 
& = \Aro \An
\end{align*}

So,
\begin{align*}
\ykn \mleq \sykn
& \Leftrightarrow
0 \mleq
\Aro \An
\begin{bmatrix}
\xrsup - \xr_{k-n} \\
\En (\Wrsup - \Wrk)
\end{bmatrix}
+ C E (\msup{\w} -\wk)
\end{align*}

We can derive the same inequality with $\iykn$:

\begin{align*}
\iykn \mleq \ykn
& \Leftrightarrow
0 \mleq
\Aro \An
\begin{bmatrix}
\xr_{k-n} - \xrinf \\
\En (\Wrk - \Wrinf)
\end{bmatrix}
+ C E (\wk - \minf{\w})
\end{align*}

If we are able to find what is the maximum $\xr_{k-n}$ for the constraint given before, then the problem is solved. I need to find the interval where all the trajectories lies for an integral constraint over the noise.

This equation correspond to an integral constraint over the noise.

\end{document}