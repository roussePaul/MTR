\documentclass{article}

\usepackage[english]{babel}
\usepackage{import}
\usepackage{enumitem}

%% MATH
\usepackage{amsmath, amsfonts, amsthm}
\usepackage{nicefrac}
\usepackage{breqn}
\usepackage{tabularx}
\usepackage{stmaryrd} % for llbracket and rrbracket
\usepackage[ruled]{algorithm2e}

% LTL temporal operator
\DeclareFontFamily{U}{matha}{\hyphenchar\font45}
\DeclareFontShape{U}{matha}{m}{n}{
      <5> <6> <7> <8> <9> <10> gen * matha
      <10.95> matha10 <12> <14.4> <17.28> <20.74> <24.88> matha12
      }{}
\DeclareSymbolFont{matha}{U}{matha}{m}{n}
\DeclareMathSymbol{\vDash}{3}{matha}{'52}
\DeclareMathSymbol{\nvDash}{3}{matha}{'50}

\DeclareMathOperator*{\argmin}{arg\,min}
\makeatletter
\newcommand{\tuple}[1]{
	\@tempcnta=0
	\def\mylist{#1}
	\@for\next:=\mylist\do{
	\ifnum \@tempcnta = 0 \else ,\linebreak[1]\fi \next
	\advance\@tempcnta\@ne}
}
\makeatother

%\renewcommand{\splitatcommas}[1]{#1}

\newcommand{\buchi}[0]{B\"uchi}

%% LTL symbols defintion
\usepackage{latexsym} % get the right \Diamond symbol
\newcommand{\LTLalways}		{\ensuremath{\square}}
\newcommand{\LTLeventually}	{\ensuremath{\Diamond}}
\newcommand{\LTLuntil}		{\ensuremath{\mathsf{U}}}
\newcommand{\LTLnext}		{\ensuremath{\bigcirc}}
\newcommand{\LTLimply}			{\ensuremath{\rightarrow}}
\newcommand{\true}			{\ensuremath{\top}}


\newcommand{\outpost}{ \ensuremath{\overline{Post} } }
\newcommand{\inpre}{ \ensuremath{\overline{Pre} } }


\newcommand{\vect}[1]{\ensuremath{ \mathbf{#1}}}
\newcommand{\leftint}{\left \llbracket}
\newcommand{\rightint}{\right \rrbracket}


%% GRAPHICS
\usepackage{graphicx}
\graphicspath{{./plots/}{./env/}}
\usepackage[mode=buildnew]{standalone}
\usepackage{epstopdf}

\usepackage{tikz}
\usetikzlibrary{shapes.geometric, arrows, positioning, calc, fit, arrows, decorations.pathreplacing}
\usepackage{pgfplots}
\usepackage{subcaption}



% DEFINITION/THEOREM ENVIRONMENTS
\newcommand{\inlinedef}[1]{\textit{#1}}

\newtheorem{theorem}{Theorem}
\newtheorem{problem}{Problem}
\newtheorem{olddefinition}{Definition}
\newtheorem{property}{Property}
\newtheoremstyle{named}{}{}{\itshape}{}{\bfseries}{:\\}{.5em}{#1 \thmnote{(#3)}}
\theoremstyle{named}
\newtheorem*{namedtheorem}{Theorem}
\newtheorem*{nameddefinition}{Definition}
\newtheorem*{namedproperty}{Property}

\newcommand{\thmsymbol}{\( \blacktriangle \)}
\newcommand{\propsymbol}{\( \blacklozenge \)}

\newenvironment{nameddef}[1]
    {\begin{samepage}
    \begin{nameddefinition}[#1]
    \renewcommand{\qedsymbol}{\thmsymbol}\pushQED{\qed}
    }
    {
    \popQED % in order to pop it before (in case of itemize) just call it in an item
    \end{nameddefinition} 
    \end{samepage}
    }
\newenvironment{namedprop}[1]
    {\begin{samepage}
    \begin{namedproperty}[#1]
    \renewcommand{\qedsymbol}{\propsymbol}\pushQED{\qed}
    }
    {
    \popQED % in order to pop it before (in case of itemize) just call it in an item
    \end{namedproperty} 
    \end{samepage}
    }
\newenvironment{namedtheo}[1]
    {\begin{samepage}
    \begin{namedtheorem}[#1]
    \renewcommand{\qedsymbol}{\propsymbol}\pushQED{\qed}
    }
    {
    \popQED % in order to pop it before (in case of itemize) just call it in an item
    \end{namedtheorem} 
    \end{samepage}
    }
\newenvironment{prop}[0]
    {\begin{samepage}
    \begin{property}
    \renewcommand{\qedsymbol}{\propsymbol}\pushQED{\qed}
    }
    {
    \popQED % in order to pop it before (in case of itemize) just call it in an item
    \end{property} 
    \end{samepage}
    }
\newenvironment{definition}[0]
    {\begin{samepage}
    \begin{olddefinition}
    \renewcommand{\qedsymbol}{\thmsymbol}\pushQED{\qed}
    }
    {
    \popQED % in order to pop it before (in case of itemize) just call it in an item
    \end{olddefinition} 
    \end{samepage}
    }
    
    
%% TRANSITION SYSTEM COMMANDS
\usepackage[all]{xy}
\usepackage{amssymb}
\newdir{|>}{-{\scriptscriptstyle{\blacktriangleright}}}
\newcommand{\transition}{
  \ensuremath{ \xymatrix{\ar@{-|>}[r]&} }
  }
\newcommand{\labelledtransition}[1]{
  \ensuremath{ \xymatrix{\ar@{-|>}[r]^{#1}&} }
  }
\newcommand{\systransition}[2]{
  \ensuremath{ \xymatrix{\ar@{-|>}[r]^{#2}_{#1}&} }
  }

\newcommand{\Post}[2]{Post_{#2}^{#1}}
\newcommand{\Reach}[2]{Reach^{#1}({#2})}
\newcommand{\ReachSeq}[2]{Reach^{*#1}(#2)}

\newcommand{\sysDef}[1]{(\)}
  
%% COMMENTS
\newcommand{\comment}[1]{\textit{\textbf{#1}}}

%% SUPER SCRIPT STUFF
\makeatletter
\newcommand\newsubcommand[3]{\newcommand#1{#2\sc@sub{#3}}}
\def\sc@sub#1{\def\sc@thesub{#1}\@ifnextchar_{\sc@mergesubs}{_{\sc@thesub}}}
\def\sc@mergesubs_#1{_{\sc@thesub#1}}

\newcommand\newsupcommand[3]{\newcommand#1{#2\sc@sup{#3}}}
\def\sc@sup#1{\def\sc@thesup{#1}\@ifnextchar^{\sc@mergesups}{^{\sc@thesup}}}
\def\sc@mergesups^#1{^{\sc@thesup#1}}
\makeatother

\newcommand{\T}[1]{#1{}^\top}
\newcommand{\proj}[1]{|_{#1}}


\begin{document}
\title{Error in partially observed abstractions}
\author{Paul Rousse}
\maketitle



\newcommand{\mle}{\prec}
\newcommand{\mleq}{\preceq}
\newcommand{\minf}[1]{\underline{#1}}
\newcommand{\msup}[1]{\overline{#1}}



\newcommand{\x}{\vect{x}}%
\renewcommand{\u}{\vect{u}}%
\newcommand{\w}{\vect{w}}%

\newcommand{\y}{\vect{y}}%
\newcommand{\yk}{\vect{y}_k}%
\newcommand{\ykn}{\vect{y}_{k+1}}%

\newcommand{\xk}{\vect{x}_k}%
\newcommand{\xkn}{\vect{x}_{k+1}}%

\newcommand{\uk}{\vect{u}_k}%
\newcommand{\wk}{\vect{w}_k}%

\newcommand{\xo}{\vect{x}^o}%
\newcommand{\xr}{\vect{x}^r}%

\newcommand{\Ao}{A_o}%
\newcommand{\Ar}{A_r}%
\newcommand{\Aro}{A_{ro}}%

\newcommand{\Bo}{B_o}%
\newcommand{\Br}{B_r}%

\newcommand{\Eo}{E_o}%
\newcommand{\Er}{E_r}%
\newcommand{\Ero}{E_{ro}}%

\newcommand{\Xr}{X_r}%

\newcommand{\Xrinv}{\mathcal{X}_r}%
\newcommand{\xrinf}{\minf{\x}_r}%
\newcommand{\xrsup}{\msup{\x}_r}%


\newcommand{\Wsup}{\msup{W}}
\newcommand{\Winf}{\minf{W}}
\newcommand{\Wk}{W_k}

\renewcommand{\wr}{\vect{w}^r}
\newcommand{\wrsup}{\msup{\vect{w}}^r}
\newcommand{\wrinf}{\minf{\vect{w}}^r}
\newcommand{\Wrsup}{\msup{W}^r}
\newcommand{\Wrinf}{\minf{W}^r}
\newcommand{\Wrk}{W^r_k}

\newcommand{\traj}{\varphi}

\newcommand{\z}{\vect{z}}%
\newcommand{\zk}{\z_k}%
\newcommand{\zkn}{\z_{k+1}}%

\newcommand{\Z}{\mathbf{z}}
\newcommand{\Zk}{\Z_k}
\newcommand{\Zkn}{\Z_{k+1}}
\newcommand{\hk}{\vect{h}_{k}}
\newcommand{\h}{\vect{h}}
\newcommand{\abs}[1]{\left|#1\right|}
\newcommand{\wnoise}{\msup{\sigma}}

\newcommand{\size}{N}
\newcommand{\norminf}[1]{\left\|#1\right\|_{\infty}}

\section{System definition}
We will take the following system:

\begin{equation}
\begin{aligned}
\xkn &=
\begin{bmatrix} \Ao&\Aro\\ 0& \Ar \end{bmatrix} \xk + 
\begin{bmatrix} \Bo\\ \Br \end{bmatrix} \uk + 
\begin{bmatrix} \Eo & \Ero \\ 0 & \Er \end{bmatrix} \wk
\\
\yk &= C\xk = \xo_k
\end{aligned}
\end{equation}

We will suppose that the system is monotonic. 
And that the noise and the inputs are bounded:
\begin{equation}
\begin{aligned}
\u &\in [\minf{\u},\msup{\u}] \subset \mathbb{R}^{n_u}\\
\w &\in [\minf{\w},\msup{\w}] \subset \mathbb{R}^{n_w}\\
\end{aligned}
\end{equation}

\section{Abstraction reduction}

The abstraction reduction is done in this way. First, we define the smallest invariant of the unobserved state:
$$
\Xrinv = [\xrinf,\xrsup]
$$
where,
$$
\begin{aligned}
\xrinf &= (I-A)^{-1} (\Br \minf{\u} + \Er \minf{\w})\\
\xrsup &= (I-A)^{-1} (\Br \msup{\u} + \Er \msup{\w})
\end{aligned}
$$

Lets now define analytically the set $\Xr(U)$ that correspond to the image of $\Xrinv$ after applying the control sequence $U = [u_0,\dots,u_{n-1}]$.
Thanks to the monotonicity property, we have the following property:
\newcommand{\xui}{\minf{\x}^U_r}
\newcommand{\xus}{\msup{\x}^U_r}
\newcommand{\xu}{\x^U_r}
$$\Xr(U) = [\xui,\xus]$$
where
$$\xui = \traj (\xrinf,U,\Winf)
\textrm{, }
\xus = \traj (\xrsup,U,\Wsup)$$
$$ \Winf = [\minf{\w},\dots,\minf{\w}]
\textrm{, }
\Wsup = [\msup{\w},\dots,\msup{\w}]$$
so that $\Wsup$ and $\Winf$ are a sequence of $n$ elements.


\newcommand{\An}{\mathcal{A}^r_n}
\newcommand{\Bn}{\mathcal{B}^r_n}
\newcommand{\En}{\mathcal{E}^r_n}
\begin{align} \label{expr:xur}
\xu &=
\begin{bmatrix}
A_r^n & A_r^{n-1} &\dots & A_r & I
\end{bmatrix}
\begin{bmatrix}
\x_0 \\
B_r \u_0 + E_r \w_0\\
\vdots \\
B_r \u_{n-1} + E_r \w_{n-1}\\
\end{bmatrix}\\
\xu &=
\An
\begin{bmatrix}
\x_0 \\
\Bn U + \En W
\end{bmatrix}
\end{align}

where
$$
\An = 
\begin{bmatrix}
A_r^n & A_r^{n-1} &\dots & A_r & I
\end{bmatrix}
\textrm{, }
\Bn =
diag( 
\begin{bmatrix}
B_r &
\dots &
B_r
\end{bmatrix})
$$
$$\textrm{ and }
\En = 
diag(\begin{bmatrix}
E_r &
\dots &
E_r
\end{bmatrix} )
$$

$$
W = \begin{bmatrix}
\w_0\\
\vdots\\
\w_{n-1}
\end{bmatrix}
\textrm{, }
U = \begin{bmatrix}
\u_0\\
\vdots\\
\u_{n-1}
\end{bmatrix}
$$

\newcommand{\xuki}{\minf{\x}^{U_k}_r}
\newcommand{\xuks}{\msup{\x}^{U_k}_r}
\newcommand{\xuk}{\x^{U_k}_r}
Let define the variable $\xuki$ and $\xuks$ (guess it from previous definition).

\section{Admissible noise}
In the previous sections, we have been using a dynamical model to build an abstraction $S_a$. In this part, we will solve the opposite problem:  we would like to find all the admissible models that can be alternatively simulated by $S_a$.

In our case, we will just investigate the set of admissible noise sequences.

\newcommand{\ANoise}{\Omega}
\newcommand{\NoiseSet}{\mathcal{W}}
\newcommand{\infseq}{\omega}
\newcommand{\xa}{\x^a}

\newcommand{\sykn}{\msup{\y}_{k+1}}%
\newcommand{\iykn}{\minf{\y}_{k+1}}%
For a sequence of noise given, if the following condition is verified for every trajectory of the system at every timesteps, then the noise sequence is an admissible noise sequence for the system:
\begin{equation} \label{valid_transition}
\ykn \in [\iykn,\sykn]
\end{equation}
where 
\begin{align*}
\sykn &= C \msup{\xo}_{k+1} = 
C A 
\begin{bmatrix}
\xo_{k} \\
\xuks
\end{bmatrix}
+ C B \uk + C E \msup{\w}
\\
\iykn &= C \minf{\xo}_{k+1} 
= C A 
\begin{bmatrix}
\xo_{k} \\
\xuki
\end{bmatrix}
+ C B \uk + C E \minf{\w}
\\
\ykn &= C \xo_{k+1}
= C A 
\begin{bmatrix}
\xo_{k} \\
\xuk
\end{bmatrix}
+ C B \uk + C E \wk
\end{align*}

We will now use the inequalities \ref{valid_transition} to get a condition over $\Wk$:
\begin{align*}
\ykn \mleq \sykn
& \Leftrightarrow
0 \mleq \sykn - \ykn \\
& \Leftrightarrow
0 \mleq
C A \begin{bmatrix} 0 \\ \xuks-\xuk \end{bmatrix}
+ C E (\msup{\w} -\wk)
\\
& \Leftrightarrow
0 \mleq
C A 
\begin{bmatrix} 0 \\ \An \end{bmatrix} 
\begin{bmatrix}
\xrsup - \xr_{k-n}\ \\
\En (\Wrsup - \Wrk)
\end{bmatrix}
+ C E (\msup{\w} -\wk)
&& \textrm{see \ref{expr:xur}}
\end{align*}

We have the following equality:
\begin{align*}
C A \begin{bmatrix} 0 \\ \An \end{bmatrix} 
& = \Aro \An
\end{align*}

So,
\begin{equation}\label{upp_equi}
\begin{split}
\ykn \mleq \sykn
& \Leftrightarrow
0 \mleq
\Aro \An
\begin{bmatrix}
\xrsup - \xr_{k-n} \\
\En (\Wrsup - \Wrk)
\end{bmatrix}
+ C E (\msup{\w} -\wk)
\end{split}
\end{equation}

We can derive the same inequality with $\iykn$:

\begin{equation}\label{low_equi}
\begin{split}
\iykn \mleq \ykn
& \Leftrightarrow
0 \mleq
\Aro \An
\begin{bmatrix}
\xr_{k-n} - \xrinf \\
\En (\Wrk - \Wrinf)
\end{bmatrix}
+ C E (\wk - \minf{\w})
\end{split}
\end{equation}

\newcommand{\filter}{\mathcal{F}}
Inequality \ref{borne_sup} and \ref{low_equi} can be seen as a filter $G$. Let $\zk=[\xr_{k-n},\wr_{k-n},\dots,\wr_{k-1}]$  defined by:
\begin{equation}
\filter:
\left\{
\begin{split}
\zkn &= F \zk + G \wk \\
\hk &= H \zk + K \wk \\
\end{split}
\right.
\end{equation}
where
\begin{equation}
F = \begin{bmatrix}
A & E & 0\\
0 & 0 & I_\z\\
0 & 0 & 0 \\
\end{bmatrix}
\textrm{, }
H = 
\Aro \An
\begin{bmatrix}
I \\
\En
\end{bmatrix}
\textrm{, }
K = \En
\end{equation}
with $I_\z$ the identity matrix of size $(n-2)\size$.
Let,
\begin{align*}
\msup{\h} &= H \msup{\z} + K \wrsup\\
\minf{\h} &= H \minf{\z} + K \wrinf
\end{align*}
Then \ref{upp_equi} and \ref{low_equi} are satisfied if the output of the filter $\filter$ is in $[\minf{\h},\msup{\h}]$ for all the timesteps.

\end{document}