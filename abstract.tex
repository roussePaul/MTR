\documentclass{article}

\begin{document}
%
%Problem of my master thesis
%
%Description of the abstraction
%Finite state controller

\begin{abstract}
%--- --- --- Intro --- --- ---
% What is an abstraction
%% finite representation of a complex system
%#### Begin directely with a finite abstraction and controller synthesis, I do not need further theory that I will not use. Just talk about the abstraction
%#### I need to begin with what I did: Formal methods applied to controller synthesis.
%## Take it slowly
%!!! Trop catégorique
% Temporal logic ~ high level specification 
% Correctness
% Tout doit apparaitre dans la premiere phrase
% LTL formal controller synthesis
Formal controller synthesis methods with temporal logic specifications try at the same time to guaranty the correctness of the high level properties with a system that might have complex behaviour.
Most often, discrete abstractions of a continuous system are used in order to reduce the size of the problem.
%% Trop general
If a behavioural relationship between the system and its abstraction is met, then a valid controller for the abstraction will be valid as well for the system.
%% used later to generate finite state controller
The finiteness of the abstraction allows us to use computer science tools to synthesize a finite state controller (graphs search and fixed point algorithms, temporal logics).
% Why do we do it
%% automated framework
These methods are automatic (the controller synthesis can be created automatically from the specifications) and offline (the controller is created beforehand).
%% high level tasks (LTL specifications)

%--- --- --- Abstraction --- --- ---
% What abstraction did I use, How did I build it
%% past input memory used to compute reachable sets, discretize in the frame of the reached sets, do not discretize unsuefull information

%In this work, we investigate abstractions with a state extended by memories of the last inputs.
%
% S -> S'
In this work, we investigate a system with a state extended by memories of the last inputs.
% S' -> Sa
These memories are then used to replace part of the state with computation of reachable sets.
% Sa -> Sa'
The final abstraction is obtained by a discretization the resulting state space.
% Sa' x C < S x C
As the abstraction is alternatingly simulated by the extended system, a controller solution of problem for the abstraction is a solution for the extended system.
% --- --- --- Algo --- --- ---
% Non determinism
%% why it is not cool to have self loops
Specifications of the controller are given in Linear Temporal Logic formula (LTL).
The product of the non deterministic automaton and the B\"uchi automaton of the LTL specification is used to find a finite state controller.
To ensure the fairness property of the solution, we had to find plans without unfair cycles.
We used a strong cyclic algorithm with a local fairness function
that indicate if the system will escape a cycle in finite time or can loop indefinitely.
%%% does not conserve the reachability properties (due to self loops)
%To overcome this, we have been using a strong cyclic planning algorithm with a fairness function. This function indicate if the system will escape a cycle in finite time or can loop indefinitely.
%% how we can benefit from self loops (self loops make model smaller)
% Strong cyclic algorithm
%
In some cases, the state extension by input memories can counteract the loss of information from the partial observation of the state.
%


% --- --- --- Experiements --- --- ---
% SML
%% Chosen model, double integrator
This controller synthesis method with Linear Temporal Logic specifications has been successfully used and tested on single and multiple quadricopters scenarios in the Smart Mobility Lab of KTH.

\end{abstract}


\end{document}