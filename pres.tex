\documentclass{beamer}
\usepackage{etex}

\usetheme{KTHprofile}

\title{Multi agent control with LTL specifications and abstraction with input memories}
\subtitle{Master thesis 2016}
\author{Paul Rousse}
\institute{Automatic Departement}
\date{\today}


\usepackage{graphicx}
\graphicspath{{./plots/}{./env/}{./results/}{./pres/}}
\usepackage[mode=buildnew]{standalone}
\usepackage{epstopdf}
\usepackage[absolute,overlay]{textpos}
\usepackage{tikz}
\usetikzlibrary{shapes,arrows,calc}
\usepackage{pgfplotstable} % For \pgfplotstableread


%% MATH
\usepackage{amsmath, amsfonts, amsthm, amssymb}
\usepackage{nicefrac}
\usepackage{breqn}
\usepackage{tabularx}
\usepackage{stmaryrd} % for llbracket and rrbracket
\usepackage[ruled]{algorithm2e}

% LTL temporal operator
\DeclareFontFamily{U}{matha}{\hyphenchar\font45}
\DeclareFontShape{U}{matha}{m}{n}{
      <5> <6> <7> <8> <9> <10> gen * matha
      <10.95> matha10 <12> <14.4> <17.28> <20.74> <24.88> matha12
      }{}
\DeclareSymbolFont{matha}{U}{matha}{m}{n}
\DeclareMathSymbol{\vDash}{3}{matha}{'52}
\DeclareMathSymbol{\nvDash}{3}{matha}{'50}

\DeclareMathOperator*{\argmin}{arg\,min}
\makeatletter
\newcommand{\tuple}[1]{
	\@tempcnta=0
	\def\mylist{#1}
	\@for\next:=\mylist\do{
	\ifnum \@tempcnta = 0 \else ,\linebreak[1]\fi \next
	\advance\@tempcnta\@ne}
}
\makeatother

%\renewcommand{\splitatcommas}[1]{#1}

\newcommand{\buchi}[0]{B\"uchi}

%% LTL symbols defintion
\usepackage{latexsym} % get the right \Diamond symbol
\newcommand{\LTLalways}		{\ensuremath{\square}}
\newcommand{\LTLeventually}	{\ensuremath{\Diamond}}
\newcommand{\LTLuntil}		{\ensuremath{\mathsf{U}}}
\newcommand{\LTLnext}		{\ensuremath{\bigcirc}}
\newcommand{\LTLimply}			{\ensuremath{\rightarrow}}
\newcommand{\true}			{\ensuremath{\top}}
\renewcommand{\and}			{\ensuremath{\wedge}}


\newcommand{\outpost}{ \ensuremath{\overline{Post} } }
\newcommand{\inpre}{ \ensuremath{\overline{Pre} } }

%% MATH COMMANDS
\newcommand{\vect}[1]{\ensuremath{ \mathbf{#1}}}
\newcommand{\leftint}{\left \llbracket}
\newcommand{\rightint}{\right \rrbracket}
\newcommand{\abs}[1]{\left|#1\right|}
\newcommand{\card}[1]{\left|#1\right|}


\usepackage{tikz}
\usetikzlibrary{shapes.geometric, arrows,automata, positioning, calc, fit, arrows, decorations.pathreplacing,patterns}
\usepackage{pgfplots}
\usepgfplotslibrary{groupplots,fillbetween}
\usepackage{relsize}


%% TRANSITION SYSTEM COMMANDS
\usepackage[all]{xy}
\usepackage{amssymb}
\newdir{|>}{-{\scriptscriptstyle{\blacktriangleright}}}
\newcommand{\transition}{
  \ensuremath{ \xymatrix{\ar@{-|>}[r]&} }
  }
\newcommand{\labelledtransition}[1]{
  \ensuremath{ \xymatrix{\ar@{-|>}[r]^{#1}&} }
  }
\newcommand{\systransition}[2]{
  \ensuremath{ \xymatrix{\ar@{-|>}[r]^{#2}_{#1}&} }
  }

\newcommand{\Post}[2]{Post_{#2}^{#1}}
\newcommand{\Reach}[2]{Reach^{#1}({#2})}
\newcommand{\ReachSeq}[2]{Reach^{*#1}(#2)}

\newcommand{\sysDef}[1]{(\)}

\newcommand{\altsim}{\preceq_{\mathcal{AS}}}
  

\newcommand{\T}[1]{#1{}^\top}
\newcommand{\proj}[1]{|_{#1}}

\renewcommand{\exp}[1]{e^{#1}}



\newcommand{\SSunobs}{\mathfrak{X}^i}
\newcommand{\SSobs}{\mathfrak{X}^o}
\newcommand{\Ninputs}{N}%
\newcommand{\y}{\vect{y}}%
\newcommand{\x}{\vect{x}}%
\newcommand{\xa}{\vect{x}^a}%
\newcommand{\xobs}{\vect{x}^o}%
\newcommand{\Xobs}{X^o}%
\newcommand{\Xobsinit}{X^o_0}%
\newcommand{\xunobs}{\vect{x}^i}%
\newcommand{\Xunobs}{X^i}%
\newcommand{\Sunobs}{\mathcal{S}^i}%
\newcommand{\Xuinv}{{\mathcal{X}^i}}%
\newcommand{\pastuseq}{\vect{u}_{n-\Ninputs},\dots,\vect{u}_{n-1}}%
\newcommand{\pastuseqn}{\vect{u}_{n+1-\Ninputs},\dots,\vect{u}_{n}}%
\newcommand{\Pastuseq}{U_n}%
\newcommand{\sys}{\mathcal{S}}%
\newcommand{\sysa}{\mathcal{S}_a}%
\newcommand{\sysaU}{\mathcal{U}}%
\newcommand{\sysA}{\mathcal{S}_A}%
\newcommand{\sysB}{\mathcal{S}_B}%
\newcommand{\sysC}{\mathcal{S}_C}%
\newcommand{\Usys}{\mathcal{U}}%
\newcommand{\Wsys}{\mathcal{W}}%
\newcommand{\Uunobs}{{\mathcal{U}^i}}%
\newcommand{\Wunobs}{{\mathcal{W}^i}}%
\newcommand{\uobs}{\vect{u}^o}%
\newcommand{\wobs}{\vect{w}^o}%
\newcommand{\uunobs}{\vect{u}^i}%
\newcommand{\wunobs}{\vect{w}^i}%
\newcommand{\Dunobs}{n^i}% Dimension of the ss of unobs
\newcommand{\R}{\mathbb{R}}% real set
%%
% Monotone systems
\newcommand{\mle}{\prec}
\newcommand{\mleq}{\preceq}
\newcommand{\minf}[1]{\underline{#1}}
\newcommand{\msup}[1]{\overline{#1}}
%%
% linear system
\newcommand{\X}{X}%
\newcommand{\Xinv}{\mathcal{X}}%
\newcommand{\U}{\mathcal{U}}%
\newcommand{\W}{\mathcal{W}}%
\newcommand{\vu}{\vect{u}}%
\renewcommand{\u}{\vect{u}}%
\renewcommand{\U}{\mathcal{U}}%
\newcommand{\w}{\vect{w}}%
\newcommand{\s}{\vect{s}}%
\newcommand{\st}{\vect{t}}%
%
% Continuous model
\newcommand{\traj}{\varphi}
%
% discretization of the abstraction
\newcommand{\xd}{\x_d}
\newcommand{\vd}{\vect{v}_d}
\newcommand{\Nobs}{N^o} % number of states of the discretization of the state space \SSobs
%
\newcommand{\dt}{dT}%

\newcommand{\red}[1]{ {\color[rgb]{1,0,0}{#1}} }
\begin{document}

\begin{frame}
\titlepage
\end{frame}

%% INTRO
\section{Introduction}

% Introduct with the project, wind turbine
%	- what do we want to achieve: motion planning 
%	- what are the specifications: collision avoidance between the agents
%	- one solution: -> using formal methods in order to do this
\begin{frame}
\frametitle{Wind turbine maintenance scenario}

\begin{figure}
\begin{minipage}{0.32\textwidth}
\centering
\includegraphics[width=\textwidth]{aerowork}
\end{minipage}
\hspace*{\fill}%
\begin{minipage}{0.35\textwidth}
\centering
\includegraphics[width=\textwidth]{inspectionquad}
\end{minipage}
\end{figure}

\only<1>{

\begin{figure}
\begin{minipage}{0.45\textwidth}
\centering
\includegraphics[width=\textwidth]{defect}
\end{minipage}
\hspace*{\fill}%
\begin{minipage}{0.45\textwidth}
\centering
\includegraphics[width=\textwidth]{repair}
\end{minipage}
\end{figure}
}
\only<2->{
\begin{itemize}
\item Motion and task specifications
\item Safety requirements (avoid collision, stay inside a safety area)
\end{itemize}
}
\only<3>{
\begin{itemize}
\item[$\Rightarrow$] Linear Temporal Logic (LTL)
\item[$\Rightarrow$] formal controller synthesis
\end{itemize}
}
\end{frame}

\begin{frame}
\frametitle{Outline}
\tableofcontents
\end{frame}


\section{Preliminaries}
% Describe how does these methods works, the different steps in order to build the controller


\newcommand{\planframe}[1]{
\begin{frame}
\frametitle{Recap}
\begin{figure}
\def\framenbre{#1}
\includestandalone[width=\textwidth]{diagram_pres}
\end{figure}
\end{frame}
}


\subsection{Temporal logic}
%info: first you talk about how you will give this specifications (to say that they will be given in a automaton form, this justify the fact that we try to find an abstractino of the system in the form of a finite transition system)

% LTL, conversion to a buchi automaton, explain what are the conditions so that a path is accepted in the buchi automaton and what it means on the LTL formula
\begin{frame}
\frametitle{High level specification: the temporal logic}
\begin{block}{Linear temporal logic (LTL)}
$$ \varphi ::= 
\true \mid 
a \mid 
\varphi_1 \land \varphi_2 \mid
\lnot \varphi \mid
\LTLnext \varphi \mid
\varphi_1 \LTLuntil \varphi_2$$
\end{block}

Temporal operators:
\begin{itemize}
\item $\LTLnext$ next
\item $\LTLuntil$ until
\item $\LTLalways$ always
\item $\LTLeventually$ eventually
\end{itemize}
\pause

\begin{figure}
\begin{minipage}{0.45\textwidth}
$\varphi = \LTLalways \LTLeventually a$
\end{minipage}
{\LARGE$\Rightarrow$}%
\begin{minipage}{0.45\textwidth}
$\sigma_1 = 111a111a\overline{111a}$\\
\red{$\sigma_2 = 111a111a111\overline{1}$}\\
\red{$\sigma_3 = 11111111111\overline{1}$}\\
\end{minipage}
\end{figure}

\end{frame}

\begin{frame}
\frametitle{\buchi{} Automaton}

\begin{block}{Nondeterministic B\"{u}chi Automaton}
$\mathcal{A}_{\varphi} = (Q, Q_0, 2^{AP}, \delta, \mathcal{F})$
where:
\begin{itemize}
\item $Q$ finite set of states;
\item $Q_0 \subset Q$ a set of initial states;
\item $2^{AP}$ the alphabet;
\item $\delta: Q \times 2^{AP} \times Q$ a transition relation ;
\item $\mathcal{F} \subseteq Q$ set of accepting states.\popQED
\end{itemize}
\end{block}

\only<2>{
$$\sigma = l_1,l_2,l_3,... \rightarrow q_0,q_1,q_2,...$$}

\only<3>{
Acceptance condition: $Inf(\{q_0,q_1,q_2,...\}) \cap \mathcal{F} \neq \emptyset$
}

\only<4>{
\begin{figure}
\begin{minipage}{0.45\textwidth}
$\sigma_1=111a111a\overline{111a}$\\
\red{$\sigma_2=111a111a111\overline{1}$}\\
\red{$\sigma_3=11111111111\overline{1}$}\\
\end{minipage}
{\LARGE$\Rightarrow$}%
\begin{minipage}{0.45\textwidth}
\includegraphics[width=1.2cm]{buchi}
\end{minipage}
\end{figure}
}

\end{frame}


\subsection{Abstraction}
% abstractions methods (unfiorm discretization)
\begin{frame}
\frametitle{Abstraction}

\begin{figure}
\centering
\begin{minipage}{.45\textwidth}
  \centering
  \includestandalone[width=\linewidth]{state_space_repr}
  \caption{Model of the robot}
\end{minipage}
{\LARGE$\rightarrow$}%
%
\only<1>{
\begin{minipage}{.45\textwidth}
$\mathcal{T}_c = (X,X_0,\U, Post_{\u},H)$
\begin{itemize}
\item $X$ states
\item $X_0 \subseteq X$ initial states
\item $U$ input set
\item $Post_{\u} \subseteq X \times U \times X$ transition relation
\item $H:X \rightarrow Y$ output map
\end{itemize}
\end{minipage}%
}%
%
\only<2>{
\begin{minipage}{.45\textwidth}
  \centering
\tikzstyle{nodestyle} = [draw,circle]
\tikzstyle{trans} = [->]
\tikzstyle{labelcont} = [pos=0.7,inner sep = 1pt]
\resizebox {0.7\columnwidth} {!} {
\begin{tikzpicture}[node distance = 1cm, auto]
    % Place nodes
    \node [nodestyle] (n1) {n1};
    \node [nodestyle,right of=n1,yshift=2.5cm,xshift=2cm] (n2) {n2};
    \node [nodestyle,below of=n2] (n3) {n3};
    \node [nodestyle,below of=n3] (n4) {n4};
    \node [nodestyle,below of=n4] (n5) {n5};
    \node [nodestyle,below of=n5] (n6) {n6};
    \node [nodestyle,below of=n6] (n7) {n7};
    
    \newcommand{\cont}{$\mathbf{u}_n$}
    
    \draw [trans] (n1) -- node[labelcont] {\cont} (n2);
    \draw [trans] (n1) -- node[labelcont] {\cont} (n3);
    \draw [trans] (n1) -- node[labelcont] {\cont} (n4);
    \draw [trans] (n1) -- node[labelcont] {\cont} (n5);
    \draw [trans] (n1) -- node[labelcont] {\cont} (n6);
    \draw [trans] (n1) -- node[labelcont] {\cont} (n7);
\end{tikzpicture}
}
\caption{Finite Transition System}
\end{minipage}
}

\end{figure}

\end{frame}



\subsection{Controller synthesis}

\begin{frame}
\frametitle{Controller synthesis}

\only<1>{
\begin{block}{Product of a NBA and a FTS}
$\mathcal{A}_p = \mathcal{T}_c \otimes \mathcal{A}_\varphi
= (Q',Q_0',\delta',\mathcal{F}')$
\begin{itemize}
\item $Q' = X \times Q$ is the set of states,
\item $Q_0' = X_0 \times Q_0$ is the set of initial states,
\item $\mathcal{F}' = X \times \mathcal{F}$ the acceptance set,
\item $\delta' \subseteq Q' \times U \times Q'$
is the transition relation defined by
$\left \langle x',q' \right \rangle \in \delta'(\left \langle x,q \right \rangle ,u)$
iff $x' \in Post_u(x)$, $q' \in \delta(q,L_c(x'))$ and 
$$\forall h \in H(Post_u(x)),\exists y \in H^{-1}(h) \cap Post_u(x), \delta(q,L_c(y)) \neq \emptyset$$
\end{itemize}
\end{block}
}

\only<2->{
Use your favourite graph search algorithm (Depth/Breadth First Search, Fixed points...)
\begin{figure}
\includestandalone[width=0.7\textwidth]{plan_alg}
\end{figure}
}

\pause
$\Rightarrow$ Controller =  $\{( (q_1,n_1),u_1),((q_2,n_2),u_2), \dots, ((q_n,n_n),u_n)\}$
\end{frame}

\planframe{0}
\planframe{1}
\planframe{2}
\planframe{3}
\planframe{5}

%% PLAN
%% Present a rough plan of what you did:
%	- abstraction
%	- algorithm used

%% ABSTRACTION
\section{State extended abstraction}
% abstraction of the quadricopter, talk about the transient state that belong to the second integrator model, the complexity of the second integrator model
% bring the solution of the input memories
% explain the steps to build the abstraction
% show that the actual admissible noise is bigger set (give the formula to get the equation and the bode diagram of it)
\newcommand{\vv}{\mathbf{v}}
\begin{frame}
\frametitle{State extended abstraction}
Dynamical system: 
\begin{itemize}
\item state: $(\x^{obs},\x^{unobs})$
\item transition: $(\x^{obs},\x^{unobs}) \systransition{S}{\u} (\x^{obs}_+,\x^{unobs}_+)$
\end{itemize}

\pause
Abstraction:
\begin{itemize}
\item $(\x^{obs},\u_{n - \Ninputs},...,\u_{n-1})$
\item transition:
$(\x^{obs},\u_{n - \Ninputs},...,\u_{n-1}) 
\systransition{S_a}{\u}
 (\x^{obs}_+,\u_{n+1-\Ninputs},...,\u_{n-1},\u)$
 \end{itemize}
\end{frame}


\begin{frame}
\frametitle{State extended abstraction}

\newcommand{\pictsize}{0.5\textwidth}
	
\only<1>{
\begin{figure}
\includestandalone[width=\textwidth]{abstraction_process_2}
\end{figure}
}

\only<2>{
\begin{figure}
	\newcommand*{\showinv}{}
	\newcommand*{\showunobstrans}{}
	\includestandalone[width=\pictsize]{abstra_proc_plots}
\end{figure}}
	
\only<3>{
\begin{figure}
\includestandalone[width=\textwidth]{abstraction_process_2}
\end{figure}
}
\end{frame}

\begin{frame}
\frametitle{Comparison with uniform discretization}

\begin{figure}
\begin{minipage}{0.4\textwidth}
\includestandalone[width=\linewidth]{state_space_repr}
\end{minipage}
\begin{minipage}{0.4\textwidth}
\includestandalone[width=\textwidth]{feasible_models_2}
\end{minipage}
\end{figure}

\only<1>{
\begin{equation*}
\begin{split}
\dot{\vv} &= k (\vv_{ref} - \vv) + \w_v\\
\dot{\x} &= \vv + \w_x\\
\varphi &= (\LTLalways \LTLeventually a) \and (\LTLalways \LTLeventually b) 
\end{split}
\end{equation*}
}

\only<2->{
\begin{itemize}
\item more conservative (the information is selected more wisely)
\end{itemize}
}
\end{frame}

\begin{frame}
\frametitle{Admissible noise}

\begin{figure}
\begin{minipage}{0.45\textwidth}
\centering
\includegraphics[width=\textwidth]{noise_1}
\caption{Dynamical system}
\end{minipage}
\pause
\begin{minipage}{0.45\textwidth}
\centering
\includegraphics[width=\textwidth]{noise_2}
\caption{Abstraction}
\end{minipage}
\end{figure}
\end{frame}	

\begin{frame}
\frametitle{Admissible noise}

$\mathcal{W}_a = \{ \{\w_n\}_{n\in\mathbb{N}}  \mid \mathcal{L}_1(G^{unobs} \ast G_{\w} ) \leq \sigma \}$

$\mathcal{W} \subset \mathcal{W}_a$

\begin{figure}
\begin{minipage}{0.45\textwidth}
\centering
\includestandalone[width=\textwidth]{bode_plots_2}
\end{minipage}
\begin{minipage}{0.45\textwidth}
\centering
\includestandalone[width=\textwidth]{speed_plots_reachables_sets_2}
\end{minipage}
\end{figure}

\end{frame}

\section{Planner}
%% ALGORITHM
% explain what was the trouble (take in account non determinism of the model)
\begin{frame}
\frametitle{Path planning algorithm}

Planner specifications:
\begin{itemize}
\pause
\item Non deterministic FTS
\pause
\item Self-loops and cycles
\pause 
\item go to the goal set in finite time
\end{itemize}

\pause
\vspace*{0.5cm}
$\Rightarrow$ Strong cyclic planner with local fairness property

\pause
\vspace*{0.5cm}
Escape time property in the cycle $\mathcal{C}$: 
$\forall \{\x_k\}_k \in \mathcal{X}^*,
\x_0 \in \mathcal{C} \Rightarrow
\exists k\in \mathbb{N},
\x_{k} \notin \mathcal{C}$


\end{frame}

% talk mabout the conditions of the buchi automaton and what is not nice with non determinism / self loops -> introduce the fairness property.

% present the plan decomposition,
\newcommand{\onlyframe}[2]{%
\only<#1>{%
\def\framenbre{#2}%
\includestandalone[width=0.5\textwidth]{plan_decomposition_beamer}%
}%
}
\begin{frame}
\frametitle{Backward Reachability Algorithm - $\varphi = \LTLeventually \LTLalways G$}
\begin{figure}
\onlyframe{1}{2}%
\onlyframe{2}{3}%
\onlyframe{3}{4}%
\onlyframe{4}{5}%
\onlyframe{5}{6}%
\onlyframe{6}{7}%
\end{figure}

\end{frame}

% say that you have created a backward reachability alogirthm that go through the solution and check if the controller configuration is fair



\section{Experiments}
\begin{frame}
\frametitle{Multi agent experiment}
\begin{itemize}
\item 0 memories (equivalent to the first integrator model)
\item $\varphi = (\LTLalways \neg out) \and (\LTLalways \LTLeventually a) \and (\LTLalways \LTLeventually b) \and (\LTLalways \neg collide)$
\end{itemize}

\begin{figure}
  \begin{minipage}{0.3\textwidth}
    \centering
    \includegraphics[width=\linewidth]{multi_ltl/multi11}
  \end{minipage} 
  \begin{minipage}{0.3\textwidth}
    \centering
    \includegraphics[width=\linewidth]{multi_ltl/multi12}
  \end{minipage} 
\end{figure}

\pause
The video !
\end{frame}

\section{Conclusion}
%% CONCLUSION
% sum up rapidely what you have done
% say what could be done

\begin{frame}
\frametitle{Tack s\r{a} mycket!}

Tack s\r{a} mycket!

Du kan fr\r{a}ga vad du vill, jag ska svara (bara om det \"ar m\"ojligt).

\end{frame}

\end{document}