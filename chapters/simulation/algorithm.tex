\section{Strong cyclic planner}
%///////////////////////////////////////////////////////////
%*** Self loops ***
%	!! pb pas vraiment à propos de la réduction, plus à propos de l'algo
%	- arguments about considering the self loops (this is about the strategy chosen for the algorithm, not about the abstraction)
%		. special case where there is no inputs: single integrator
%			¤ why do I choose the single integrator?
%			-> I have to choose a model, this case is interesting to show that it is really efficient in some case
%		. give the link between state space discretization and sampling rate
%		. way smaller discretization for the same sampling rate
%		%% show the graphs about the single integrator
%
%	- disadvantage of self loops
%		-> create asynchronous systems that are terrible for multi agent systems, we are loosing time information which is super bad for synchronisation of course
%		can I find a number to talk about it
%///////////////////////////////////////////////////////////
%
\newcommand{\Snosl}{S_{nosl}}
\newcommand{\Ssl}{S_{sl}}
\newcommand{\xdsl}{\xd^{sl}}
\newcommand{\xdnosl}{\xd^{nosl}}

\subsection{Self loops}
In this part we intend to show how does the self loops can reduce the complexity of the formal controller synthesis.
We will work with the 0-input memory abstraction ($\Ninputs=0$) that can be treated as the single integrator model.

Let $\Ssl$ the abstraction with self loops. And $\Snosl$ the abstraction without any self loops (except on null control inputs).
The sampling time $\dt$ for the 2 models will be the same.
\comment{which model are you talking about?}

For the abstraction $\Ssl$, the only requirement so that the abstraction is controllable is that $\abs{\u}>\abs{\w}$ for all non null controls $\u$ and all noise $\w$.
This inequality assert that the system must be able to counteract the influence of the noise.
In the case of the abstraction $\Snosl$, the control input must at least bring the state to the next cell.
Let $\xdnosl$ the discretization step of $\Snosl$, $\xdnosl$ must verify $\xdnosl<\abs{\u + \w}\dt$ for any non null control actions $\u$ and any noise $\w$. 

Figure \ref{fig:simple_int_sl} show that for a fixed length of the environment, the model $\Ssl$ will have the same complexity for any noise level (as long as it is smaller than the input $\u$).
In the case of the abstraction $\Snosl$, as the number of cells needed to find a solution to the controller is inversely proportional to the noise magnitude.

\begin{figure}
	\includestandalone[width=0.5\textwidth]{single_int_self_loops_less}
	\caption{Testing environment}
	\label{fig:simple_int_sl}
\end{figure}

\subsection{Size of the fixed points}
%///////////////////////////////////////////////////////////
%*** Size of the fixed point ***
%	%% plot the influence of the size of the fixed point for the single integrator
%	- show that the best option is to limit it as possible (but it does worth it for small number of fixed points)
%///////////////////////////////////////////////////////////
%
The real drawback of strong cyclic planning method lie in the search of a fair control combination. Lets try to see what are the limitation of the algorithm.
In the previous experiments, we have been investigating cases where the maximal fixed point size is equal to 2. However, this complexity is increasing exponentially in the number of controls in $\U$ and in the number of possible successors (the number of combinations of controls for a set of $n$ nodes is equal to $|\U|^n$).

The current implementation of our algorithm is not able to deal with computation of fixed point bigger that 9 (which corresponds in our model where $|\U|= 9$ to $|\U|^9 \approx 3.8e8$ possible fixed points configurations).
We have been running a test (see figure \ref{fig:cycle_sizes}) with increasing number of successors. 
The more successors we have, the biggest will be the fixed points of the plan. This result in a higher complexity.

Models with undesirable self loops (self loops that appear on input different from \uo) produce systems where a transition will be taken after an unknown finite time.
This hide all the time information of our system.
For multi agent applications, these models behave the same as asynchronous system.
This result in a possible state space that is generally too high to use any graph search algorithm on it.

\begin{figure}
\centering
\begin{minipage}[t]{0.49\linewidth}
	\includestandalone[width=\linewidth]{nbre_succ_single_int}
	\caption{Influence of the strong cycles sizes.}
	\label{fig:cycle_sizes}
\end{minipage}
\begin{minipage}[t]{0.49\linewidth}
	\includestandalone[width=\linewidth]{nbre_node_single_int_size_fp}
	\caption{Influence of the number of successors over the number of nodes for different models: without self loops (wol), with self loops (wl), with 9 or 4 successors.}
	\label{fig:max_cycle_tree}
\end{minipage}
\end{figure}
