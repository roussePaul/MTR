\subsection{Noise}
%///////////////////////////////////////////////////////////
%*** Noise *** (abstr)
%	%% presentation of what apply to our model:
%		. give the values of the constraints on the noise for dbl1 for a chosen sampling rate, introduce this with a curve of an admissible noise sequence
%		. admissible noise set not bigger for a constant input: show a corve of the maximum noise set
%	
%	%% influence of the number of inputs:
%		. plot some curves for 1, 2 or 3 inputs memories
%	
%	%% can I link it to the sampling rate (ie plot the bode diagrams according to the sampling rates)
%		. plot the maximum oscillation or the max peak for the noise according to the sampling rate
%		. link between this and the computation of the invariants
%		. acceptable error is higher at the beginning the same on infinite time
%///////////////////////////////////////////////////////////
%
\comment{Draw different trajectories for the noise}
\comment{Show a trajectory where it is actually working}
\comment{Draw Bode diagram if I have some}

\comment{It would be nice to have a temporal interpretation of it instead of a discrete one (as I have been varying the sampling rate)}

In order to show how does the admissible noise behave according to the system, we have been plotting the bode diagram for several models (see figure \ref{fig:bode_gain}).
When the sampling time is much faster than the dynamic (case $k=0.5$), the approximation of the reachable sets is worst than when the dynamic is faster than the discrete time (case $k=2.0$). This over approximation is necessary to admit all the possible noises. However, this over approximation make it possible to use noises of higher magnitude. These noise will likely be of high frequency.
Please note that the maximum static noise admissible is the same for all the models and is the same than the second integrator abstraction without input extended state.

\begin{figure}
\includestandalone[width=\linewidth]{plots/bode_plots}
\caption{Bode plots of the admissible noise for different models.}
\label{fig:bode_gain}
\end{figure}


