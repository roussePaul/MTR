\section{Noise}
%///////////////////////////////////////////////////////////
%*** Noise *** (abstr)
%	%% presentation of what apply to our model:
%		. give the values of the constraints on the noise for dbl1 for a chosen sampling rate, introduce this with a curve of an admissible noise sequence
%		. admissible noise set not bigger for a constant input: show a corve of the maximum noise set
%	
%	%% influence of the number of inputs:
%		. plot some curves for 1, 2 or 3 inputs memories
%	
%	%% can I link it to the sampling rate (ie plot the bode diagrams according to the sampling rates)
%		. plot the maximum oscillation or the max peak for the noise according to the sampling rate
%		. link between this and the computation of the invariants
%		. acceptable error is higher at the beginning the same on infinite time
%///////////////////////////////////////////////////////////
%
\comment{Draw different trajectories for the noise}
\comment{Show a trajectory where it is actually working}
\comment{Draw Bode diagram if I have some}

\comment{It would be nice to have a temporal interpretation of it instead of a discrete one (as I have been varying the sampling rate)}


When the sampling time is much faster than the dynamic, the reachable are over approximated. This over approximation admit more noise (see figure \ref{fig:bode_gain}).
The maximum static noise admissible is the same for all the model and is the same than the not reduced abstraction.

\begin{figure}
\includestandalone[width=\linewidth]{plots/bode_plots}
\caption{Bode plots of the admissible noise for different models.}
\end{figure}


