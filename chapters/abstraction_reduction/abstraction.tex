\newcommand{\Cont}{\mathcal{C}}%
%
% Introduction of the necessity of reduction of an abstraction
Graph search algorithm complexities are dependant on the size of the abstraction.
These methods suffer from the state space explosion problem: the complexity of graph search algorithm have usually an exponential complexity in the size of the state space.

% Property that needs to be verified by the abstraction
Therefore it is of a great interest to design small abstractions for the controller synthesis.
The new abstraction needs to verify alternating similarity relationship with the original abstraction.

The term "reduction" will be justified later. But this does not always correspond to a reduction of the abstraction size (the set of all the input sequences might be larger than the size of the states dimensions that have be suppressed). However, the efficient utilizations of this method will correspond to a reduction of the state space size.

\section{Related work}
Talk about formal verifications methods.

The needs of discrete abstraction.

%% MOVE THIS PART
In \cite{tabuada2009verification}, the link between in hybrid systems is investigated. 
\begin{nameddef}{System}\label{def:system}
$S = (X,X_0,\U, \systransition{S}{}, Y,H)$
where:
\begin{itemize}[noitemsep,nolistsep]
\item $X$ is a set of states;
\item $X_0 \subset X$ a set of initial states;
\item $\U$ a set of inputs;
\item $\systransition{S}{} \subseteq X \times \U \times X$ a transition relation ;
\item $Y$ a set of outputs;
\item $H:X \rightarrow Y$ an output map.\popQED
\end{itemize}
\end{nameddef}

Note: in the future we will use a second equivalent notations for the transition relation $\systransition{S}{\u}$ of a system $S$: $\Post{S}{\u}$  defined for $\x \in X$ and $\u \in \U$ by:
\begin{equation}
\Post{S}{\u}(\x) = \{ \x \in X \mid \exists \x' \in X, \x \systransition{S}{\u} \x' \}
\end{equation}

\comment{This system can model both discrete and continuous systems, which make it a good candidate in order to deal with hybrid systems.}

The alternating simulation relation between 2 systems is the link between discrete abstraction and the continuous representation of this system.
For comoditiy, the definition of alternative simulation is rewritten here (\cite{tabuada2009verification}):
\begin{nameddef}{Alternating simulation} \label{def_alt_sim}
Let $\sysA$ and $\sysB$ 2 systems with $Y_a=Y_b$, $\sysA$ is alternatingly simulated by $\sysB$ if there exists a relation $R \subseteq X_a \times X_b$ that verify:
\begin{enumerate}[noitemsep,nolistsep]
\item $\forall x_{a0} \in X_{a0}, \exists x_{b0} \in X_{b0}, (x_{a0},x_{b0}) \in R$
\item $\forall (x_a,x_b) \in R, H_a(x_a) = H_b(x_b)$
\item $\forall (x_a,x_b) \in R, \forall u_{a} \in \U_{a}, \exists u_{b} \in \U_{b}$\\
$\forall x_b' \in \Post{\sysB}{u_b}(x_b),\exists x_a' \in \Post{\sysA}{u_a}(x_a), (x_a',x_b') \in R$
\popQED
\end{enumerate}
\end{nameddef}
The alternating simulation relation between $\sysA$ and $\sysB$ is weaker than the bisimulation relation (that require the alternating simulation relation between $\sysA$ and $\sysB$ and between $\sysB$ and $\sysA$).

Lets denote the composition of systems by the operator $\times$.
If $\sysA$ alternatingly simulate by $\sysB$ for a controller $\Cont$ composable with $\sysA$ or $\sysB$, then $\sysA \times \Cont$ verify the same reachability properties than $\sysB \times \Cont$, and $\sysB \times \Cont$ verify the same safety properties than $\sysA \times \Cont$.


%% INPUT SEQUENCE REACHABLE SETS
In the next part we will focus on systems with inputs memory. The knowledge of unobserved variables will be replaced by the knowledge of an input sequence $\Pastuseq$.

For a system $S$, let $\Reach{S}{U} \subseteq X$ defined for a finite sequence $U \in \U^\star$ of $N \in \mathbb{N}$ control actions by:
\begin{equation}
\begin{split}
\x \in \Reach{S}{U}
\Leftrightarrow &
\exists \x_0 \in X_0,
\forall i<N, \exists \x_i \in X,\\
&\x_0 \systransition{S}{\u_0} \x_1
\systransition{S}{\u_1} \dots
\systransition{S}{\u_{N-1}} \x
\end{split}
\end{equation}
$\Reach{S}{U}$ correspond to all the reachable states with the sequence inputs $U$.

%% Define the reached states:
\renewcommand{\v}{\vect{v}}
\newcommand{\useq}{\v_{1-n},\dots,\v_{0}}
\begin{definition}
For a finite sequence $\Pastuseq = \left\{ \useq \right\}$ of $n$ controls in $\U$,
let $X(\Pastuseq) \subseteq X$ the set of all the states reached by a control sequence terminating with $\Pastuseq$:
\begin{equation}
\ReachSeq{S}{\Pastuseq}
=
\bigcup_{\{\vu_i\}_{i \le \infty} \in \U^\star}
\Reach{S}{\{\vu_0,\vu_1,\dots,\useq\}}
\end{equation}
\end{definition}

\section{Introduction}
We will first introduce the abstraction used, then treat the case of dynamical systems, then with linear systems.


\section{Abstraction reduction} \label{sec:abstraction}
\comment{Questions: Which system to choose and when? Idea, I begin with the definition of the system to introduce the reached sets, give the property of alterning simulation relation for this type of system and then go for dynamical systems, talk about the reached sets.}%
\comment{Define the reached sets.}%
\comment{Do I want to talk about an abstraction or a system?}%
\comment{How can it reduce the size if it is a continuous system?}%
\comment{Partir d'un system discret}%
%
\comment{I need to create the abstraction with memories first.}%
%
Let the system $\sys = (\tuple{X,X_0,\U,\transition,Y,H})$
and the same system extended with a memory of the last $\Ninputs$ control actions $\sys'$ defined by
$\sys' =  (\tuple{X',X_{0}',\U,\transition,Y',H'})$ 
where:
\begin{itemize}[nolistsep,noitemsep]
\item $X' = X \times \U^{\Ninputs}$ the set of states, 
\item $X_{0}' = X_0 \times \U^{\Ninputs}$ the set of initial states,
\item $Y' = Y \times \U^{\Ninputs}$ the set of outputs,
\item $H'$ the output map defined for all $(x,\Pastuseq) \in X'$ as $H'(x') = (H(x),\Pastuseq)$
\item and the transition relation defined by:
\begin{equation}
\begin{split}
(\x,\u_{n - \Ninputs},...,\u_{n-1}) 
\systransition{S'}{\u} &
 (\x',\u_{n+1-\Ninputs},...,\u_{n-1},\u)\\
\Longleftrightarrow 
&
\left\{
\begin{split}
&\x \systransition{S}{\u} \x' \\
&\x \in \ReachSeq{S}{\u_{n - \Ninputs},...,\u_{n-1}}
\end{split}
\right.
\end{split}
\end{equation}
\end{itemize}

In this section we will design an abstraction $\sysa$ that alternately simulate the system $\sys'$.

% Describe how it is done
In control synthesis we would use abstraction in order to design a controller that will be used afterward by the original system. If the system and its abstraction verify a property of bisimulation, then all the property of the abstraction composed with the controller hold for the original system composed with the same controller. In other words, the bisimulation relation guaranty that reachability and safety properties are the same.
However, most of the time this relation is too demanding.
In our case, we will just guaranty the alternating simulation relation between $\sys$ and $\sysa$ which conserve the reachability properties between 2 systems.
\comment{Talk with Pierre Jean about this part.}

We will assume that the state $\x$ of the system $\sys$ can be decomposed in this way $\displaystyle\T{\x} = \T{[\T{\xobs},\T{\xunobs}]}$ where $\xobs \in Y$ can be observed and $\xunobs$ is an unobserved (internal) state.
Lets call $\SSunobs$ the subspace of $\xunobs$ and $\SSobs$ the one of $\xobs$.
The states of the reduced abstraction $\sysa$ will be expressed by $\xa_n = [\xobs_n,\Pastuseq]$ where $\Pastuseq = [\pastuseq]$ correspond to the $\Ninputs$ last control actions.


Lets define the set $\Xunobs(\Pastuseq) = \Reach{S}{\Pastuseq} \proj{\SSunobs}$
Literally, $\Xunobs(\Pastuseq)$ correspond to all the states that can be reached with control sequences terminating with $\Pastuseq$.
We can now "replace" the knowledge of the state  $\xunobs$ by the set of all the possible states $\Xunobs(\Pastuseq)$ after applying $\Pastuseq$.
% Definition of the reduced system
Let the system
$\sysa =  (X_a,X_{a 0}, \sysaU, \transition, Y_a, H_a)$ 
where:
\begin{itemize}[nolistsep,noitemsep]
\item $X_a = \Xobs \times \sysaU^{\Ninputs}$ the set of states, 
\item $X_{a 0} = \Xobsinit \times  \mathcal{U}^{\Ninputs}$ the set of initial states,
\item $Y_a = \Xobs \times \U^\Ninputs$ the set of outputs,
\item $H_a$ the output map that correspond to the projection from $X_a$ on $\Xobs$.
\item and the transition relation is defined by:
\begin{equation}
\begin{split}
(\xobs_n,\vect{u}_{n - \Ninputs},...,\vect{u}_{n-1}) 
\labelledtransition{\vu} 
& (\xobs_{n+1},\vect{u}_{n+1-\Ninputs},...,\vect{u}_{n-1},\vu)\\ \Longleftrightarrow 
\xobs_{n+1} \in 
& H_a(Post^S_{\vu}(\{\xobs\} \times \Xunobs(\vect{u}_{n - \Ninputs},...,\vect{u}_{n-1}))
\end{split}
\end{equation}
\end{itemize}

%% MISSING PROOF THAT IT IS AN ALTERNATING SIMULATION RELATION
\begin{prop}
\comment{I have been messing with the alternate simulation relation, I need to inverse all of them}
$\sysa$ is alternatingly simulated by $\sys'$.
\end{prop}

\begin{proof}
Let $R$ the relation defined by:
\begin{equation}
R = \{ (\x',\xa) \in X' \times X_a \mid H'(\x') = H_a(\xa) \}
\end{equation}
By definition of the systems $S'$, $\sysa$ and of the relation $R$, conditions 1 and 2 of definition \ref{def_alt_sim} are already verified.

In order to prove that condition 3 lie, it is sufficient to show that the unobserved part of $\x$ is contained in the set $\Xunobs(\Pastuseq)$.
As we know that for a all $\xunobs$,$ \xunobs \in \Xunobs(\Pastuseq)$, this imply that there ...

Let $(\x',\xa) \in R$, $\u \in U$ and $\x'_+ \in \Post{S'}{\u}(\x')$.
As $H'(\x') = H_a(\xa)$, $\x' \proj{\U^\Ninputs} = \xa \proj{\U^\Ninputs}$, we will denote this quantity with $\Pastuseq$.
By definition of $\Xunobs(\Pastuseq)$,
we know that $\x' \proj{\SSunobs} \in \Xunobs(\Pastuseq)$,
so
$\x' \in \{\x' \proj{\SSobs} \} \times \Xunobs(\Pastuseq) \times \Pastuseq$
which imply that
$\x'_+ \in \Post{S'}{\u}(\{\x' \proj{\SSobs} \} \times \Xunobs(\Pastuseq) \times \Pastuseq )$.
By taking $\xa_+ = H_a(\x'_+) \in \Post{S_a}{\u}(\xa)$,
we have $(\x'_+,\xa_+) \in R$.
\end{proof}

\section{Dynamical systems}
\comment{Add the dynamical representation of the system def.}

As the definition \ref{def:system} of a system can be used in order to represent continuous and discrete dynamical systems.
In this case, the computation of the sets $\Xunobs(\Pastuseq)$ is a reachability problem.
It can be solved in 2 steps: find the smallest invariant $\Xuinv$ of $\sys$ dynamics on $\xunobs$; compute $\Xunobs(\Pastuseq)$ the image of $\Xuinv$ after applying a control sequence that terminate with $\Pastuseq$.

\begin{figure}
\centering
%\includegraphics[width=0.5\linewidth]{invariant_set}
\caption{The image of the invariant set $\Xuinv$ after applying the control sequence $\Pastuseq$ is $\Xunobs(\Pastuseq)$}
\end{figure}

% Here I am trying to explain why I have been choosing a smaller class of systems
Interesting case scenarios happen when the knowledge about $\Xunobs(\Pastuseq)$ is preferable to the actual measure of the state $\xunobs$.
If $\Xunobs(\Pastuseq)$ is unbounded it can results in smaller abstraction (it can possibly create an infinite number of successors).
We will therefore focus on systems that have bounded invariant sets $\Xuinv$.

From now we will use linear time invariant systems. We use them mainly for their ease of manipulation.