\newcommand{\xo}{\vect{x}^o}%
\newcommand{\xr}{\vect{x}^r}%
%
\newcommand{\Ao}{A_o}%
\newcommand{\Ar}{A_r}%
\newcommand{\Aro}{A_{ro}}%
%
\newcommand{\Bo}{B_o}%
\newcommand{\Br}{B_r}%
%
\newcommand{\Eo}{E_o}%
\newcommand{\Er}{E_r}%
\newcommand{\Ero}{E_{ro}}%
%
\newcommand{\Xr}{X_r}%
%
\newcommand{\Xrinv}{\mathcal{X}_r}%
\newcommand{\xrinf}{\minf{\x}_r}%
\newcommand{\xrsup}{\msup{\x}_r}%
%
%
\newcommand{\Wsup}{\msup{W}}%
\newcommand{\Winf}{\minf{W}}%
\newcommand{\Wk}{W_k}%
%
\renewcommand{\wr}{\vect{w}^r}%
\newcommand{\wrsup}{\msup{\vect{w}}^r}%
\newcommand{\wrinf}{\minf{\vect{w}}^r}%
\newcommand{\Wrsup}{\msup{W}^r}%
\newcommand{\Wrinf}{\minf{W}^r}%
\newcommand{\Wrk}{W^r_k}%
%
\newcommand{\z}{\vect{z}}%
\newcommand{\zk}{\z_k}%
\newcommand{\zkn}{\z_{k+1}}%
%
\newcommand{\xkn}{\x_{k+1}}%
\newcommand{\xk}{\x_{k}}%
\newcommand{\uk}{\u_{k}}%
\newcommand{\wk}{\w_{k}}%
\newcommand{\yk}{\y_{k}}%
\newcommand{\ykn}{\y_{k+1}}%
%
\newcommand{\Z}{\mathbf{z}}%
\newcommand{\Zk}{\Z_k}%
\newcommand{\Zkn}{\Z_{k+1}}%
\newcommand{\hk}{\vect{h}_{k}}%
\newcommand{\h}{\vect{h}}%
\newcommand{\wnoise}{\msup{\sigma}}%
%
\newcommand{\size}{N}%
\newcommand{\norminf}[1]{\left\|#1\right\|_{\infty}}%
%
\newcommand{\xui}{\minf{\x}^U_r}%
\newcommand{\xus}{\msup{\x}^U_r}%
\newcommand{\An}{\mathcal{A}^r_n}%
\newcommand{\Bn}{\mathcal{B}^r_n}%
\newcommand{\En}{\mathcal{E}^r_n}%
\newcommand{\xuki}{\minf{\x}^{U_k}_r}%
\newcommand{\xuks}{\msup{\x}^{U_k}_r}%
\newcommand{\xuk}{\x^{U_k}_r}%
%
\newcommand{\ANoise}{\Omega}%
\newcommand{\NoiseSet}{\mathcal{W}}%
\newcommand{\infseq}{\omega}%
%
\newcommand{\sykn}{\msup{\y}_{k+1}}%
\newcommand{\iykn}{\minf{\y}_{k+1}}%
%
\newcommand{\filter}{\mathcal{F}}%
%
\section{Error sensitivity}
%
In the previous sections, we have been using a dynamical model to build an abstraction $S_a$. In this part, we will solve the opposite problem:  we would like to find all the admissible models that can be alternatively simulated by $S_a$.
In our case, we will find all the admissible models with the same dynamic as the initial model $S$ but with a different noise set.

As we do not observe $\xunobs$, the error on $\SSunobs$ does not have a direct effect on the input extended state abstraction.
However, it does have an impact as the dynamics on $\SSunobs$ influence the one on $\SSobs$.
In this part, we will see that the admissible noise set is actually bigger (in the sense of inclusion) than the noise set used for the modellisation.

We will consider the system definition \ref{lin_sys_decoupled} with noise sets and input sets 2 monotone sets expressed by:
\begin{equation}
\begin{aligned}
\U &= [\minf{\u},\msup{\u}]\\
\W &= [\minf{\w},\msup{\w}]\\
\end{aligned}
\end{equation}

%% Explain what is the definition of an admissible noise
A sequence of noise is admissible if it does not violate the transitions of the abstraction for any possible trajectory.
In our case, this mean that for any sequence of noise, the observation of the next state must belong to the monotone interval of observations.
This can be expressed in this way:
\begin{equation} \label{valid_transition}
\ykn \in [\iykn,\sykn]
\end{equation}
where 
\begin{align*}
\sykn &= C \msup{\x}_{k+1} = 
C A 
\begin{bmatrix}
\xo_{k} \\
\xrsup
\end{bmatrix}
+ C B \uk + C E \msup{\w}
\\
\iykn &= C \minf{\x}_{k+1} 
= C A 
\begin{bmatrix}
\xo_{k} \\
\xrinf
\end{bmatrix}
+ C B \uk + C E \minf{\w}
\\
\ykn &= C \x_{k+1}
= C A 
\begin{bmatrix}
\xo_{k} \\
\xr
\end{bmatrix}
+ C B \uk + C E \wk
\end{align*}

We will now use the inequalities \ref{valid_transition} to get conditions over the noise:
\begin{align}
\ykn \mleq \sykn
& \Leftrightarrow
0 \mleq \sykn - \ykn
\nonumber \\
& \Leftrightarrow
0 \mleq
C A \begin{bmatrix} 0 \\ \xrsup-\xr_k \end{bmatrix}
+ C E (\msup{\w} -\wk)
\nonumber \\
& \Leftrightarrow
0 \mleq
\Aro ( \xrsup-\xr_k ) + C E (\msup{\w} -\wk)
\label{equi_upp}
\end{align}
We would like to have a condition over the noise only. However, the process $\xr_k$ depend on the input $\uk$. Therefore, we will introduce the variable $\zk$ defined by:
\begin{equation}
\zkn = \Ar \zk + \Er \wk + \Br \msup{\u}
\end{equation}
Thanks to the monotone property and to $\uk \mleq \msup{\u}$, for any trajectory if $\x_0 \mleq \z_0$, then:
\begin{equation}\label{process_ineq}
\xk \mleq \zk
\end{equation}
Lets define $\zk^* = \zk + (\Ar-I)^{-1} \Br \msup{\u}$, for all $k\in \mathbb{N}$:
\begin{equation}
\zkn^* = \Ar \zk^* + \Er \wk
\end{equation}

Then by using the inequality \ref{process_ineq} and equivalence \ref{equi_upp}, we have the following implication:
\begin{equation}
\Aro (\zk^* - (\Ar-I)^{-1} \Br \msup{\u} ) + CE \wk 
\mleq \Aro \xrsup + CE \msup{\w}
\Rightarrow 
\ykn \mleq \sykn
\end{equation}

\newcommand{\sig}{\sigma}
\newcommand{\ssup}{\msup{\sig}}
\newcommand{\sinf}{\minf{\sig}}
Finally, we have that:
\begin{equation}
\Aro \zk^* + CE \wk 
\mleq 
\ssup
\Rightarrow 
\ykn \mleq \sykn
\label{impl_upp}
\end{equation}
with $\ssup = \Aro ( \xrsup + (\Ar-I)^{-1} \Br \msup{\u} ) + CE \msup{\w}$

The same implication can be done with the lower bound of $\yk$:
\begin{equation}
\Aro \zk^* + CE \wk 
\mleq 
\sinf
\Rightarrow 
\ykn \mleq \sykn
\label{impl_low}
\end{equation}
with $\sinf = \Aro ( \xrinf + (\Ar-I)^{-1} \Br \minf{\u} ) + CE \minf{\w}$

We will express these 2 implications with a filter inequality.
Let the following filter $\filter$:
\begin{equation}
\filter:
\left\{
\begin{split}
\zkn^* &= \Ar \zk^* + \Er \wk \\
\hk &= \Aro \zk + CE \wk \\
\end{split}
\right.
\end{equation}
Then implications \ref{impl_upp} and \ref{impl_low} are satisfied if the output of the filter $\filter$ is in $[\sinf,\ssup]$ for all the timesteps.
The set $\W_a$ of noise sequences that verify:
\begin{equation}
\forall k \in \mathbb{N}, \h_k \in [\sinf,\ssup]
\end{equation}
is a subset of the admissible noise sequences.


By linearity of the filter, we know that if the sequences $\{\wk^1\}_{k\in \mathbb{N}}$ and  $\{\wk^2\}_{k\in \mathbb{N}}$ are admissible, then the sequence defined by $\{a\wk^1 + b\wk^1\}_{k\in \mathbb{N}}$ with $a,b>0$ and $a+b = 1$ is an admissible noise sequence.
Lets now consider the bode $G$ function of the filter $\filter$ and the set $\W'$ defined by:
$$\W_G = 
\{ 
\{G(w) cos(w k + \phi)\}_{k \in \mathbb{N}} 
\mid \phi \in [0,2\pi], w \in [0,2\pi] \}
$$
We have that the convex polyhedra defined by the set $\W_G$ is a subset of the admissible noise set.
This result give us a practical criteria over the set of admissible noise sequences.
It will also help us to understand some experiments.

Please note that the maximum constant noise remains unchanged.
However, noises of higher frequencies can have greater magnitudes than the one modelled initially. 


\comment{Add property that the convex polyhedra of the sinusoidal signals with magnitude less than $u$ correspond to the set of all signals so that $u_n<u$ for all $n$.}