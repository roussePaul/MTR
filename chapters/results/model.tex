\section{Model of the quadricopter}
Each quadricopter is modelled as a 2 axis double integrator with velocity feedback.

\subsection{Model of one axis}
We will use the second integrator model with a discrete set of controls.

\begin{equation}\label{eqn:lin_sys}
\begin{split}
\dot{\mathbf{x}} &= A \mathbf{x} + B \mathbf{u} + \mathbf{w}\\
\mathbf{y} &= C\mathbf{x}
\end{split}
\end{equation}
with
\begin{equation*} \label{eqn:sec_int}
A = \begin{bmatrix}
0 & 1\\ 
0 & -k
\end{bmatrix}
\textrm{, }
B = \begin{bmatrix}
0 \\ 
k 
\end{bmatrix}
\textrm{, }
k \in \mathbb{R}_+^\star
\end{equation*}

Two models will be used. The first one is the "raw" abstraction (without any extended state) where we observe both the position and the velocity of the agent:
\begin{equation}
C_{dble} = \begin{bmatrix}
1 & 0\\
0 & 1
\end{bmatrix}
\end{equation}
and the input extended state abstraction where only the position will be used:
\begin{equation}
 C_r = \begin{bmatrix}
 1 & 0
 \end{bmatrix}
\end{equation}.


As we will work on models with different sampling rates, the assumptions will be verified for the continuous system.
The discrete system sampled at $\dt$ is expressed by:
\begin{equation}
\x_{k+1} = A_d \x_k + B_d \u_k + E_d \w_k
\end{equation}
with
\begin{equation*} \label{eqn:sec_int_disc}
A = \begin{bmatrix}
1 & (\exp{-k \dt} - 1)\frac{1}{k}\\ 
0 & \exp{-k\dt}
\end{bmatrix}
\textrm{, }
B_d = \begin{bmatrix}
\exp{-k \dt} - 1 \\ 
k \exp{-k \dt}
\end{bmatrix}
\textrm{, }
E_d = Ad
\end{equation*}

We can then identify the different matrices used for the computation of the input extended state abstraction:\\
\begin{tabular}{lll}
$A_r =  \begin{bmatrix} \exp{-k\dt} \end{bmatrix} $
&
$A_z =  \begin{bmatrix} 1 \end{bmatrix} $
&
$A_{rz} =  \begin{bmatrix} (\exp{-k \dt} - 1)\frac{1}{k} \end{bmatrix}$
\\
$B_z =  \begin{bmatrix} \exp{-k \dt} - 1 \end{bmatrix} $
&
$B_r =  \begin{bmatrix} k \exp{-k\dt} \end{bmatrix} $
&
\\
$E_r =  \begin{bmatrix} \exp{-k\dt} \end{bmatrix} $
&
$E_z =  \begin{bmatrix} \exp{-k \dt} - 1 \end{bmatrix} $
&
$E_{rz} =  \begin{bmatrix} (\exp{-k \dt} - 1)\frac{1}{k} \end{bmatrix}$
\end{tabular}

We will choose an input set $\U = \{-0.2,0.0,0.2\}$.
We have been modelling the noise with $\W = [\minf{\w},\msup{\w}]$.
\begin{equation}
\minf{\w} = \T{ \begin{bmatrix} -0.02&-0.15 \end{bmatrix} }
\textbf{, }
\msup{\w} = \T{ \begin{bmatrix}  0.02& 0.15 \end{bmatrix} }
\end{equation}

The input and noise sets are bounded. The position and the velocity are not coupled and the dynamic on the velocity is asymptotically stable.
We can therefore compute the input extended state abstraction. More over, the unobserved system is monotonic.

\subsection{Discretization of the environment}
All the discretization that will be used are uniform discretization of the observed state space.
For experiments over 1 axes, we will discretize the environment in position with a discretization step $\xd$. In the case of the raw double integrator model, the velocity subspace will be discretized every $\vd$ steps.
For 2D environment, we will use the same discretization steps ($\xd$) for the 2 axis and for the 2 velocity subspaces ($\vd$).
We will call $\Nobs$ the number of states of the discretized state space $\SSobs$.

\begin{figure}
\centering
\begin{minipage}[b]{0.49\linewidth}
\includestandalone[width=\textwidth]{grid_environment}
\end{minipage}
\begin{minipage}[b]{0.49\linewidth}
\includestandalone[width=\textwidth]{state_space_repr}
\end{minipage}
\caption{Discretization of the 2D environment (left) and of the second integrator model (1D).}
\end{figure}

We will use the following notation:
\begin{equation}
\begin{split}
\um = -0.2
\textrm{, }
\up = 0.2
\end{split}
\end{equation}
