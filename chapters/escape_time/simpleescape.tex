\section*{Notations}

Let $\llbracket a,b \rrbracket = \left \{n \in \mathbb{N} \mid a \leq n \leq b \right \}$, $\left [ a,b \right ]= \left \{x \in \mathbb{R} \mid a \leq n \leq b \right \}$, $\mathbb{R}_+$ the positive subset of $\mathbb{R}$, $\mathbb{R}_+^\star = \mathbb{R}_+ \setminus \{0\}$.
Let $\left \| X \right \|$ the maximum distance of 2 points in the set $X$ ( $\left \| X \right \| = \sup_{a,b \in X} \left \| a-b\right \|$).
For a set $X$, $X^{\mathbb{N}}$ is an infinite sequence in $X$, for $x$ in $X^{\mathbb{N}}$ and $k$ in $\mathbb{N}$, $x_k$ is the $k^{\textrm{th}}$ element of the sequence $x$.
The Minkowski sum of $A,B \in X$ noted $A+B$ is the set $\left \{a+b \mid a \in A, b\in B \right \}$.
For $r>0$, let $\mathcal{B}(r) = \left \{ x \in \mathbb{R}^n \mid \left \| x \right \| \leq r \right \}$ the closed ball of radius $r$.
For $a,b$ 2 vectors of the euclidian space build on $\R^n$ for $n\in \mathbb{N}$, we will denote by $a \cdot b$ the scalar product of $a$ and $b$.

\section{Introduction}
The escape time function used to prove the fairness property is saying whether or not the system will leave a set of cells in a finite time.
This escape property must be deduced from the dynamical system, this is a complementary information to the FTS.
In this part we present one possible escape property for monotone systems.

In the section \ref{sec_escape}, we will lever a practical criteria over control actions of a single integrator system to check if the state can escape a subset of the state space in a finite time.
Then (section \ref{sec_ext_escape}) we will extend the criteria to the noisy case, and in a more general case to a monotone dynamical system.
And finally (section \ref{sec_timing}) we will integrate the notion of escaping time to use this criteria in frameworks with time constraints such as Metric Interval Temporal Logic (MITL).

\section{Escaping proof}\label{sec_escape}
In this section we will derive a condition over the control sequence of a first integrator system to check if the system can escape a bounded subset of the state space in a finite time.

In the previous sections, we have been using a controller that is respectively applying a control action $v_1,v2,...,v_n$ in region $X_1,X_2,...,X_n$ ($n \in \mathbb{N}$). Hereby, we will perform all the proofs over the $X_0 = X_1 \cup X_2 \cup ... \cup X_n$ with controls randomly chosen $\{v_1,v2,...,v_n\}$.
All the behaviours of the first system are included in the second system (as defined in \cite{tabuada2009verification}). This also means that any property valid for any trajectories of the second system will be valid for any trajectories of the first system. The following escape time function is therefore an upperbound function.

Let the system $S$:
\begin{equation}
S:\left \{
  \begin{tabular}{ll}
  $x_{k+1}$ &$= x_k + u_k$\\
  $x_k$ & $\in \mathbb{R}^N$\\
  $x_0$ & $\in X_0$\\
  $u_k$ & $\in \mathcal{U}$
  \end{tabular}
\right.
\end{equation}
with $k \in \mathbb{N}$ the timestep, $N \in \mathbb{N}$ the state space dimension, $X_0$ a bounded subset of $\mathbb{R}^N$, and $\mathcal{U} = \left \{v_1,...,v_n \right \}$ a finite set of control inputs in $\mathbb{R}^N$, let $n$ the cardinality of $\mathcal{U}$.
Let $\mathcal{X}_S \subset (X \times \mathcal{U})^{\mathbb{N}}$ the set of all the possible trajectories of $S$.

\begin{definition}
A trajectory $(x,u)$ of a system $S$ escapes $X_0$ in finite time if there exists $k$ in $\mathbb{N}$ so that $x_k \in X_0$ and $x_{k+1} \notin X_0$.
\end{definition}

\begin{property} \label{prop_escape}
For a trajectory $(x,u) \in \mathcal{X}_S$ 
$$
\exists \alpha \in \mathbb{R}^N, \forall k \in \mathbb{N}, u_k \cdot \alpha > 0
\Rightarrow
(x,u) \textit{ escape } X_0 \textit{ in a finite time} 
$$
\end{property}

\begin{proof}
Let $(x,u)$ a trajectory of the system $S$.
Let assume that there exists $\alpha$ in $\mathbb{R}^N$ so that $\forall k \in \mathbb{N}, u_k \cdot \alpha > 0$.
By computing the scalar product of the sequence $\{x_i \}_{i \in \mathbb{N}}$, with $\alpha$ and thanks to the strict inequality of the scalar product, we will show that the state diverges.

Let $k \in \mathbb{N}$, we have $x_{k} = x_0 + \sum_{i=0}^{k} u_i$.
By taking the scalar product of the previous expression and $\alpha$, we have $x_k \cdot \alpha = x_0 \cdot \alpha + \sum_{i=0}^{k} u_i \cdot \alpha$.
 
Let $E = \left \{ u_k \cdot \alpha \mid k \in \mathbb{N} \right \}$, by assumption $E$ is bounded by 0, so $\epsilon  = \inf E$ exists. As $\mathcal{U}$ is closed and bounded, $E$ is closed and bounded as well, so $\epsilon$ is a minimum and belong to $E$, so $\epsilon > 0$.

By taking the scalar product of the state sequence, we have $x_k \cdot \alpha \geq x_0 \cdot \alpha + k \, \epsilon$, so $\lim_{k \rightarrow + \infty} x_k \cdot \alpha = +\infty$ this imply that $\lim_{k \rightarrow + \infty} \left \| x_k \right \| = +\infty$ (this implication can be proved by a \textit{reductio ad absurdum}).
As $X_0$ is bounded, there exists a timestep $k_0$ for which $x_{k_0} \notin X_0$.
\end{proof}

For $V \subset \mathbb{R}^N$:
$$F(V) = \left \{ \alpha \in \mathbb{R}^N| \forall v \in V, v \cdot \alpha >0 \right \}$$

The property \ref{prop_escape} can be expressed in this way: for a trajectory $(x,u) \in \mathcal{X}_S$ with a control sequence chosen in $\mathcal{U}_t \subseteq \mathcal{U}$, if $F(\mathcal{U}_t) \neq \emptyset$ then the state $x_k$ escape $X_0$ in a finite time.

\section{Extension to convex input sets}\label{sec_ext_escape}

In our case, the system $S$ is not sufficiently expressive, 
we aim to use this model to abstract models that have an asymptotic single integrator behaviour.
The differences between the two systems will be modelled with noise.

Let the system $S'$:
\begin{equation}
S':\left \{
  \begin{tabular}{ll}
  $x_{k+1}$ &$= x_k + u_k + w_k$\\
  $x_k$ & $\in \mathbb{R}^N$\\
  $x_0$ & $\in X_0$\\
  $u_k$ & $\in \mathcal{U}$\\
  $w_k$ & $\in \mathcal{W}$
  \end{tabular}
\right.
\end{equation}
with $k \in \mathbb{N}$ the timestep, $N \in \mathbb{N}$ the state space dimension, $X_0$ a bounded subset of $\mathbb{R}^N$, and $\mathcal{U} = \left \{v_1,...,v_n \right \}$ a finite set of controls inputs in $\mathbb{R}^N$, $n$ the cardinality of $\mathcal{U}$ and $\mathcal{W} \subset \mathbb{R}^N$ the set of noise.

Contrary to the previous section, the difference between 2 consecutive states does not belong to a finite subset of controls inputs but in the set $\mathcal{U} + \mathcal{W}$.
If $\mathcal{U} + \mathcal{W}$ is a closed and bounded set then the results of Property \ref{prop_escape} can be applied to get equivalent results.

\begin{property} \label{prop_escape_noise}
For a trajectory $(x,u) \in \mathcal{X}_{S'}$ 
\begin{equation}
\begin{split}
\exists \alpha \in \mathbb{R}^N,
\forall k \in \mathbb{N},
\forall v \in \{u_k\} + \mathcal{W}, v \cdot \alpha > 0 \\
\Rightarrow
(x,u) \textit{ escape } X_0 \textit{ in a finite time} 
\end{split}
\end{equation}
\end{property}

\begin{proof}
Same proof as for Property \ref{prop_escape}. As by assumption the set $\mathcal{U} + \mathcal{W}$ is bounded and closed we can find an $\epsilon$ that minor all the scalar product of the elements of $\{u_k\} + \mathcal{W}$ for $k$ in $\mathbb{N}$.
\end{proof}

In the case of the single integrator with a finite input set, the escaping property is usable in practice because we have to check a finite number of control inputs.
For the single integrator with noise we have to check the property over an infinite amount of values: this is not algorithmically doable.
In order to reduce the complexity of the problem, we will use a convex bounded polyhedra set for $\mathcal{W}$.

Any points in a convex bounded polyhedra set can be expressed as a linear combination of the vertices: if $\mathcal{P}$ is a convex bounded polyhedra with a set $\mathcal{V}$ of vertices, then for any $x$ in $\mathcal{P}$, there exist $n_{\mathcal{V}} = \left | \mathcal{V} \right |$ real numbers $t_i \in [0,1]$ for $i \in \llbracket 1,n_{\mathcal{V}} \rrbracket$ that verify $\sum_{i=1}^{n_{\mathcal{V}}} t_i = 1$, so that $x = \sum_{i=1}^{n_{\mathcal{V}}} t_i \, v_i$.

Moreover, the convexity property conserve the scalar product inequality: for $u,v,\alpha$ in $\mathbb{R}^N$, if  $u \cdot \alpha > 0$ and $v \cdot \alpha > 0$, then for all $t$ in $[0,1]$, $ (t \, u + (1-t) \, v ) \cdot \alpha> 0$.
In the case of the convex bounded polyhedra $\mathcal{P}$ with a set $\mathcal{V}$ of vertices: if for all $v \in \mathcal{V}$ the scalar product with $\alpha$ is positive, then the scalar product of all points inside $\mathcal{P}$ is positive. This reduce the size of the problem to a finite number of points.

The previous paragraph showed that $\forall \alpha \in \mathbb{R}^N, \alpha \in F(\mathcal{V}) \Rightarrow \alpha \in F(\mathcal{P})$, so $F(\mathcal{V}) \subseteq F(\mathcal{P})$. As $\mathcal{V} \subset \mathcal{P}$, $F(\mathcal{P}) \subseteq F(\mathcal{V})$. Therefore, $F(\mathcal{V}) = F(\mathcal{P})$.

Lets consider a sequence $\{u_k\}_{k \in \mathbb{N}}$ of control chosen in  $\mathcal{U}_t \subseteq \mathcal{U}$.
As $\mathcal{U}_t + \mathcal{W} = \bigcup_{ u \in \mathcal{U}_t}\left ( \left \{ u \right\} + \mathcal{W} \right )$ where each $\left \{ u \right\} + \mathcal{W}$ is a convex bounded polyhedra, we would like to compute the set $F$ of an union of two sets.
Let $A$ and $B$ two subsets of $\mathbb{R}^N$, and $\alpha \in\mathbb{R}^N$:
\begin{equation}\label{Funion}
\begin{aligned}
  \alpha \in F(A) \cap F(B) & \Leftrightarrow \forall x \in A,x \cdot \alpha > 0 \textit{ and } \forall x \in B,x \cdot \alpha > 0
  \\ & \Leftrightarrow \forall x \in A \cup B,x \cdot \alpha > 0
  \\ & \Leftrightarrow \alpha \in F(A \cup B)
\end{aligned}
\end{equation}

so $F(A \cup B) = F(A) \cap F(B)$.

Lets $\mathcal{V}^t_i$ the set of all the vertices of the convex bounded polyhedra $\left \{ v_i \right \} + \mathcal{W}$, thanks to the definition of convex sets and from the fact that a convex bounded polyhedra can be expressed as a finite sum of its vertices, we have:

\begin{align*}
F(\mathcal{U}_t + \mathcal{W}) &= F(\bigcup_{i\in \llbracket 1,n^t \rrbracket} \left \{ \left \{ v_i \right \} + \mathcal{W} \right \} ) \\
&= \bigcap_{i\in \llbracket 1,n^t \rrbracket} F(\left \{ v_i \right \} + \mathcal{W}) && \text{(thanks to equivalence \ref{Funion})}\\
&= \bigcap_{i\in \llbracket 1,n^t \rrbracket} F(\mathcal{V}^t_i) \\
&= F( \bigcup_{i\in \llbracket 1,n^t \rrbracket} \mathcal{V}^t_i) \\
\end{align*}

The set $\bigcup_{i\in \llbracket 1,n^t \rrbracket} \mathcal{V}^t_i$ is a finite subset of $\mathbb{R}^N$ therefore, it can be used as a practical criteria.

It follows that for any trajectories $(x,u)$ of the system $S'$ with controls chosen in $\mathcal{U}_t \subseteq \mathcal{U}$, $F( \bigcup_{i\in \llbracket 1,n^t \rrbracket} \mathcal{V}^t_i) \neq \emptyset$ implies that the state $x_k$ will escape $X_0$ in a finite time.

\section{Monotone system extention}\label{monotone}
We introduced the escape time function for the single integrator model. This is somehow restrictive. We will in this section extend this result for a wider class of systems by building an alternating simulation relation between them and the single integrator system.
Let the monotone system defined by:
\begin{equation}
S:\left\{
\begin{split}
\x_{k+1} &= f(\x_k,\u_k,\w_k)\\
\x_0 &\in X_0\\
\u_k &\in \U\\
\w_k &\in \W\\
\end{split}
\right.
\end{equation}
with $\U$ finite and bounded, and $\W$ bounded and closed.

By creating the new system defined by:
\begin{equation}
S':\left\{
\begin{split}
\x_{k+1} &= \x_k + \u_k' + \w_k'\\
\x_0 &\in X_0\\
\u_k' &\in \U'\\
\w_k' &\in \W'\\
\end{split}
\right.
\end{equation}
with
$$\U' = \{ g(\x,\u) \mid \u \in \U, \x \in X_0\}$$,
$$g(\x,\u) = \frac{1}{2}(f(\x,\u_k,\minf{\w})+f(\x,\u_k,\msup{\w})) -\x$$
and 
$$\W' = \bigcup_{\x \in X_0,\u \in \U} [f(\x,\u,\minf{\w})-g(\x,\u),f(\x,\u,\msup{\w})-g(\x,\u)]$$.
As $X_0$, $\U$ and $\W$ are bounded by assumption. And if we assume that $f(X_0,\U,\W)$ is a bounded set as well, then $\U'$ and $\W'$ are bounded. $\U$ is finite, so $\U'$ is finite as well. $\W$ and $\X_0$ are closed, the function $h(\x,\u) = f(\x,\u,\w) - g(\x,\u)$ is continuously differentiable on $X_0\times\U$, so the set $\W'$ is closed as well.

Thanks to all these property, we can apply the previous results to the system $S'$. 
We should also note that the system $S'$ is alternatingly simulated by the system $S$. All the behaviour of $S$ are included in the behaviours of $S'$. This means that if the system $S'$ escape $X_0$ in finite time, then the system $S$ escape the set $X_0$ in finite time as well.

\section{Timing  information}\label{sec_timing}
In this section, we will show how to use the previous results to get an upper bound of the escaping time of $S$ trajectories starting in $X_0$.
This can be used to generate controllers with time specifications (such as in MITL frameworks).

The escaping property asserts that if all control actions of a trajectory are oriented in a preferential direction, the system will always escape the initial subset.
If we manage to find the minimum control action that is applied in a direction at each iteration, then we can find an upper bound of the iteration needed to escape the initial set.

\begin{property}
Let $\mathcal{U}_t \subseteq \mathcal{U}$ a subset of controls so that $F(\mathcal{U}_t) \neq \emptyset$. Let $\mathcal{U}_t = \{v_1^t,...,v_{n_t}^t\}$ with $n_t$ in $\mathbb{N}$.
There exists 
$ u_m = \argmin_{u \in C_{\mathcal{U}_t}} \left \| u \right \|$ so that $u_m \neq 0$ with
\begin{equation}
C_{\mathcal{U}_t}=
\left \{
\sum_{i=1}^{n_t} k_i \, v^t_i
\mid 
\forall (k_1,...,k_{n_t}) \in [0,1]^{n_t},
\sum_{i=1}^{n_t} k_i = 1,
\right \}
\end{equation}



All trajectories with control inputs chosen in $\mathcal{U}_t$  of system $S$ starting in $X_0$ escape $X_0$ in less than $T_{escape} =\frac{\left \| X_0 \right \|}{\left \| u_m \right \|}$ timesteps.
\end{property}

\begin{proof}
Firstly we will show the existence of $T_{escape}$, then that for all trajectories of the system $S$, the escaping time from $X_0$ is smaller than $T_{escape}$.
We will study all the trajectories generated by the control sequences chosen in $\mathcal{U}_t^{\mathbb{N}}$.

Intuitively $C_{\mathcal{U}_t}$ represents the temporal mean of the control sequences that we can apply with controls chosen in $\mathcal{U}_t$.
We will try to find what is the smallest control (in term of norm) that we can apply over an infinite sequence of control.
This control will give us information about the slowest escape from $X_0$ of our system for a chosen subset of controls.

Let $\mathcal{K} =  \left \{ k \in \left[0,1\right]^n \mid \sum_{i=1}^n k_i = 1 \right \}$. $\mathcal{K}$ is a closed bounded set.
Let $f_{\mathcal{U}_t}(\mathbf{k}) = \sum_{i=1}^n k_i \, v^t_i$ for $\mathbf{k} \in \mathcal{K}$, as $f$ is continuous and bounded by the closed ball $\mathcal{B} \left( \sum_{u \in {\mathcal{U}_t}} \left \| u \right \| \right)$ on $\mathcal{K}$, $C_{\mathcal{U}_t} = f_{\mathcal{U}_t}(\mathcal{K})$ is bounded and closed.

Since $C_{\mathcal{U}_t}$ is a bounded and closed set, and $\left \| \cdot \right \|$ is continuous, the image of $C_{\mathcal{U}_t}$ by $\left \| \cdot \right \|$ is a bounded closed set as well.
Therefore  $\min_{  u \in C_{\mathcal{U}_t}} \left \| u \right \|$ exists and is reached in $C_{\mathcal{U}_t}$. Let $u_m$ an element of $C_{\mathcal{U}_t}$ so that $\left \| u_m \right \| = \min_{  u \in C_{\mathcal{U}_t}} \left \| u \right \|$.

As $F(\mathcal{U}_t) \neq \emptyset$, $0 \notin \mathcal{U}_t$ and $0 \notin \mathcal{K}$ so $0 \notin C_{\mathcal{U}_t}$.
Therefore $u_m \neq 0$ and $\left \| u_m \right \| \neq 0$.
We can then define the finite number:
\begin{equation}
T_{escape} = \frac{\left \| X_0 \right \|}{\left \| u_m \right \|}
\end{equation}
where $\left \| X_0 \right \|$ is the maximum distance of 2 points taken in $X_0$.

Now, we need to prove that for any control sequence in $\mathcal{U}_t^{\mathbb{N}}$ applied in $X_0$, the state will escape $X_0$ it $T \leq T_{escape}$.

As $F(\mathcal{U}_t) \neq \emptyset$, for all trajectories $(x,u)$ with controls chosen in $\mathcal{U}_t$ the state will escape the subset $X_0$, so there exist a $T$ in $\mathbb{N}$ where $x_{T} \in X_0$ and $x_{T+1} \notin X_0$. Let define $\bar{u}$:
\begin{equation}
\bar{u} = \frac{1}{T} \sum_{i=0}^{T-1} u_i
\end{equation}
we have $x_T = x_0 + T \, \bar{u}$. $\bar{u}$ is an element of $C_{\mathcal{U}_t}$ so $\left \| \bar{u} \right \| \geq \left \| u_m \right \|$, and by definition $\left \| x_0 - x_T \right \| \leq \left \| X_0 \right \|$, therefore $T \leq T_ {escape}$.
\end{proof}

\section{Conclusion}
We have been introducing an escape time property valid a wide class of systems: monotone system. This is however a rough over-approximation and it might be that the state leave the set of considered cell in finite time but the ecape time property do not see it (even for the single integrator model).
Thanks to this function, we are able to ensure the fairness property despite the presence of self-loops in the final system.

The fairness property only check the finiteness of infiniteness of trajectories in a set of cells. We will see in the next sections that such situations could have been solved by a more precise discretization of the state space by trying to avoid any self-loops in the FTS (this have further complexity implication that we will develop later). Between these 2 approaches, the escape time property one do not provide time information any more.
However, we have been providing an upper bound of the time. This allows to use MITL frameworks.